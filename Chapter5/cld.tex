\chapter{Reduced-order Dynamics of Closed-loop Multi-body Systems}
Dynamics plays crucial role in the design, control and simulation of various multi-body systems. In order to study the behaviour of multi-DOF haptic interfaces, simulations were performed in Chapter 4 considering them as typical multi-body systems. Note that, whereas a stable and efficient dynamics algorithm provides accurate and faster simulation, an unstable and inefficient one leads to unrealistic behaviour of the systems under study. Inverse dynamics analyses on the other hand is helpful in teaching dynamics associated with the mechanisms. Various mechanical systems such as vehicles, spacecraft's, robots, haptic devices, etc. are built with complex mechanisms whose dynamics can be taught using haptics. In this chapter, a new reduced-order methodology for both inverse and forward dynamic analyses of closed-loop multi-body systems is proposed. In both the approaches towards inverse and forward dynamics analyses, elegant and recursive algorithms are obtained compared to what exists in the literature. 

In general, a closed-loop multi-body system possesses relatively fewer degrees-of-freedom (DOF) compared to the number of connected bodies. In other words, some of the joints in the system are actuated or active, whereas the rest are unactuated or passive. In the proposed methodology, a given closed-loop system is first divided into an equivalent open architecture by introducing cuts at appropriate joints. In the resulting serial and tree-type subsystems, cut joints are replaced by unknown constraint forces, also referred to as Lagrange multipliers. Such subsystem-level treatment allows one to use the already well-established algorithms for serial and tree-type systems. Note that for each subsystem, the governing equations of motion can be derived using the concept of Decoupled Natural Orthogonal Complement (DeNOC) matrices and the Lagrange multipliers associated to the cut-joints that belong to that open chain \citep{chaudhary2008dynamics,shah2013dynamics}.

For the inverse dynamics analyses, the torques at the active joints are found for a given motion of the system. In this, the Lagrange multipliers are found first by combining the equations of motion for passive links followed by joint torques. Alternatively, in forward dynamics where the system motion is obtained for a given set of active joint torques, Lagrange multipliers are calculated sequentially at subsystem level and later treated as external forces to the resulting serial or tree-type systems. The proposed methodology enables utilization of existing dynamics algorithms for open-loop systems.

In the following sections, a generalized procedure for deriving the equations of motion for the closed-loop system is presented, followed by the proposed inverse and forward dynamics algorithms along-with numerical illustrations.
\begin{figure}[t]
	\begin{center}
		\subfloat[A hydraulic escavator, \citet{janssen2005multiloop} ]{\label{fig:c5escavatorpic}\includegraphics[scale=0.6]{Chapter5/figures/escavator2.pdf}}\\ 
		\subfloat[Kinematic diagram of (a)]{\label{fig:c5escavator_systm_closed}\includegraphics[scale=0.5125]{Chapter5/figures/closed_loop.pdf}} \hspace{1.5pt}
		\subfloat[Subsystems after introducing cuts in (b)]{\label{fig:c5escavator_subsystm_open}\includegraphics[scale=0.4575]{Chapter5/figures/subsystems.pdf}} 
	\end{center}
	\caption{A multi-closed-loop system before and after introduction of cuts}
	\label{fig:c5cut_strategy}
\end{figure}
\section{Equations of Motion for a Closed-loop System}
\label{c5EOM}
In general, a closed-loop system is divided into an equivalent open-loop serial and/ or tree-type systems, referred to as subsystems, by making cuts at appropriate joints and introducing Lagrange multipliers thereby \citep{bae1987recursive,nikravesh1993systematic,blajer1994projective,shabana2013dynamics}. Figure \ref{fig:c5escavatorpic} depicts a skeletal diagram of an excavator (a multi-closed-loop system) used for debris removal in heavy industries \citep{janssen2005multiloop}. Its kinematic diagram is shown in Fig. \ref{fig:c5cut_strategy}. This 3-DOF system is actuated at joints 4, 10 and 13. The system has four closed-loops namely loop 1 or joints 1-2-15-6-1, loop 2 or joints 7-14-13-12-7, loop 3 or joints 3-4-5-8-7-15-3 and loop 4 or joints 9-10-11-2-1-9. In order to convert such multi-loop system into an equivalent open-loop system, i.e., the spanning tree, the closed kinematic loops are cut at joints, as indicated in Fig. \ref{fig:c5escavator_systm_closed}. The resulting open-loop architecture is shown in Fig. \ref{fig:c5escavator_subsystm_open} that typically consists of one or more serial and/or tree-type subsystems. The system in \ref{fig:c5escavator_systm_closed} has two serial subsystems, namely, I and III, and one tree-type subsystem, namely II. In the next step, dynamic formulation of a general tree-type system is employed to these subsystems where the Lagrange multipliers are identified as external wrenches (moments and forces) to the equations of motion.
\subsection{Equations of motion of a tree-type subsystem}
\label{c5eomtreess}
Using the notations for tree-type systems proposed in \citet{chaudhary2008dynamics}, first the unconstrained Newton-Euler (NE) equations of motion are written. The independent generalized equations of motion for the tree-type system are then derived by pre-multiplying the above NE equations with the transpose of the corresponding de-coupled natural orthogonal complement (DeNOC) matrices associated to it, as detailed later.

Note that, to derive the equations of motion for the closed-loop system, it is assumed to have \emph{v} subsystems, serial and tree-type, and \emph{n} bodies. Each subsystem in the following is represented with a subscript \emph{j}, where $j = \{1, ..., v\}$, whereas the total number of bodies in the \emph{j}$^{th}$ subsystem is denoted as \emph{n}$_j$. If the subsystem is a tree-type, it is assumed to have \emph{b} branches represented with a superscript \emph{h}, where $h = \{1,  ..., b\}$. Moreover the total number of bodies in each branch of the \emph{j}$^{th}$ tree-type subsystem is represented as \emph{n}$_j^h$. Hence, the total number of bodies in the closed-loop system under study is given by \emph{n} = $\sum\limits_{j=1}^{v}\sum\limits_{h=1}^{b}n_j^h$.\\
The unconstrained equations of motion for the \emph{j}$^{th}$ tree-type subsystem are now expressed as
\begin{equation}
\label{eqn:indp_eom}
\textbf{M}_j\dot{\textbf{t}}_j 
+ \mathbf{\Omega}_j\textbf{M}_j\textbf{E}_j\textbf{t}_j = \textbf{w}_j 
\end{equation}
where the 6\textit{n}$_j$-dimensional vectors \textbf{t}$_j$, $\dot{\textbf{t}}_j$ and \textbf{w}$_j$ are the generalized twist, twist-rate and wrench, respectively, associated to $j^th$ subsystem which are defined as \textbf{t}$_j$ $\equiv$ [(\textbf{t}$_{j}^{1}$)$^T$...(\textbf{t}$_{j}^{h}$)$^T$...(\textbf{t}$_{j}^{b}$)$^T$]$^T$, $\dot{\textbf{t}}_j$ $\equiv$ [($\dot{\textbf{t}}_{j}^{1}$)$^T$...($\dot{\textbf{t}}_{j}^{h}$)$^T$...($\dot{\textbf{t}}_{j}^{b}$)$^T$]$^T$ and \textbf{w}$_j$ $\equiv$ [(\textbf{w}$_{j}^{1}$)$^T$...(\textbf{w}$_{j}^{h}$)$^T$...\\(\textbf{w}$_{j}^{b}$)$^T$]$^T$. Moreover, the 6\textit{n}$_j^h$-dimensional twist, twist-rate and the wrench associated with branch \emph{h} of the \emph{j}$^{th}$ tree-type subsystem are defined as \textbf{t}$_j^h$ $\equiv$ [\textbf{t}$_1^T$...\textbf{t}$_k^T$...\textbf{t}$_{n_{j}^{h}}^T$]$^T$, $\dot{\textbf{t}}_j^h$ $\equiv$ [$\dot{\textbf{t}}_1^T$...$\dot{\textbf{t}}_k^T$...$\dot{\textbf{t}}_{n_{j}^{h}}^T$]$^T$ and \textbf{w}$_j^h$ $\equiv$ [\textbf{w}$_1^T$...\textbf{w}$_k^T$...\textbf{w}$_{n_{j}^{h}}^T$]$^T$.
The 6-dimensional twist, twist rate and the wrench associated to a single rigid link are defined as \textbf{t}$_k \equiv$ [$\bm{\upomega}_k^T$ $\textbf{v}_k^T]^T$, $\dot{\textbf{t}}_k \equiv$ [$\dot{\bm{\upomega}}_k^T$ $\dot{\textbf{v}}_k^T]^T$ and \textbf{w}$_k \equiv$ [\textbf{n}$_k^T$ \textbf{f}$_k^T$]$^T$ --- $\bm{\upomega}_k$, \textbf{v}$_k$, \textbf{n}$_k$ and \textbf{f}$_k$ being the 3-dimensional angular velocity, linear velocity, moment and force applied on the $k^{th}$ link about its origin \textbf{o}$_k$ as shown in Fig. \ref{fig:c5singlelinkdiag}. Please note that the subscript \emph{k} is used to indicate individual rigid link belonging to a particular branch. The matrices \textbf{M}$_j$, \textbf{\ohm}$_j$ and \textbf{E}$_j$ are the $6n_j\times6n_j$ generalized matrices of the extended mass, angular velocity and coupling term for the \emph{j}$^{th}$ subsystem having $n_j$ links, where $n_j$ = $\sum\limits_{h=1}^{b}n_j^h$, respectively. The matrices \textbf{M}$_j$, $\mathbf{\Omega}_{j}$ and \textbf{E}$_{j}$ are defined as
\begin{eqnarray}
\textbf{M}_j \equiv diag[\textbf{M}_{j}^{1}..\textbf{M}_{j}^{h}..\textbf{M}_{j}^{b}], \hspace*{2pt}\mathbf{\Omega}_j \equiv diag[\mathbf{\Omega}_{j}^{1}..\mathbf{\Omega}_{j}^{h}..\mathbf{\Omega}_{j}^{b}], \hspace*{2pt}\textbf{E}_j \equiv diag[\textbf{E}_{j}^{1}..\textbf{E}_{j}^{h}..\textbf{E}_{j}^{b}]
\end{eqnarray}
where the 6${n_{j}^{h}}\times$6${n_{j}^{h}}$ generalized matrices of the extended mass, angular velocity and coupling term for the branch \emph{h} of \emph{j}$^{th}$ subsystem are similarly defined as \textbf{M}$_j^h$ $\equiv$ $diag[\textbf{M}_{1}..\textbf{M}_{k}..\textbf{M}_{n_{j}^{h}}]$, $\mathbf{\Omega}_j^h \equiv diag[\mathbf{\Omega}_{1}..\mathbf{\Omega}_{k}..\mathbf{\Omega}_{n_{j}^{h}}], \textbf{E}_j^h \equiv diag[\textbf{E}_{1}..\textbf{E}_{k}..\textbf{E}_{n_{j}^{h}}]$. Matrices \textbf{M}$_k$, \textbf{\ohm}$_k$ and \textbf{E}$_k$ are the $6\times6$ extended mass, angular velocity and coupling matrices for the $k^{th}$ link respectively, given as
\begin{equation}
\label{eqn:MkOkEk}
\textbf{M}_k \equiv 
\left[ \begin{array}{cc}
\textbf{I}_k & \textit{m}_k\mathbf{\tilde d}_k\\
-\textit{m}_k\mathbf{\tilde d}_k & \textit{m}_k\textbf{1}\\
\end{array} \right],\hspace*{5pt}
\textbf{\ohm$_k$} \equiv 
\left[ \begin{array}{cc}
\bm{\tilde\upomega}_k & \textbf{O} \\
\textbf{O} & \bm{\tilde\upomega}_k \\
\end{array} \right],\hspace*{5pt}
\textbf{E}_k \equiv 
\left[ \begin{array}{cc}
\textbf{1} & \textbf{O} \\
\textbf{O} & \textbf{O} 
\end{array} \right]
\end{equation}
\begin{figure}[t!]
	\begin{center}
		\includegraphics[scale=0.75]{Chapter5/figures/singlelinkdiag.pdf}
		\caption{$k^{th}$ link representation}
		\label{fig:c5singlelinkdiag} % Give a unique label
	\end{center}
\end{figure}
\\In Eq. \ref{eqn:MkOkEk}, $\bm{\tilde\upomega}_k$ and $\mathbf{\tilde d}_k$ are the $3\times3$ skew-symmetric matrices associated with the 3-dimensional vectors, $\bm{\upomega}_k$ and \textbf{d}$_k$, respectively, where \textbf{d}$_k$ is the vector from the origin of the \emph{k}$^{th}$ link to its mass centre as shown in Fig. \ref{fig:c5singlelinkdiag}. Also, \textbf{I}$_k$ and \textit{m}$_k$ are the $3\times3$ inertia tensor about the origin of the \textit{k}$^{th}$ link and the mass of the \textit{k}$^{th}$ link, respectively. Note that the wrench \textbf{w}$_k$ is composed of \textbf{w$_k^E$}, the wrench due to externally applied moments and forces, \textbf{w$_k^C$}, the wrench due to the constraint moments and forces at the uncut joints, and \textbf{w$_k^\lambda$}, the wrench due to Lagrange multipliers at the cut joints, i.e., \textbf{w}$_k \equiv$ \textbf{w$_k^E$} + \textbf{w$_k^C$} + \textbf{w$_k^\lambda$}. Hence, Eq. \ref{eqn:indp_eom} can also be rewritten for the $j^{th}$ tree-type subsystem as
\begin{equation}
\label{eqn:uncons_eom}
\textbf{M}_j\dot{\textbf{t}}_j + \mathbf{\Omega}_j\textbf{M}_j\textbf{E}_j\textbf{t}_j = \textbf{w}^E_j + 
\textbf{w}^C_j + \textbf{w}_j^\lambda
\end{equation}
where \textbf{w$^E_j$}, \textbf{w$^C_j$} and \textbf{w$^\lambda_j$} denote the 6\textit{n}$_j$-dimensional vectors corresponding to the \emph{j}$^{th}$ tree-type subsystem. The generalized twist for the tree-type subsystem having \emph{n}$_j$-links is then written in terms of the joint-rates belonging to that branch as
\begin{equation}
\label{eqn:twist}
\textbf{t}_j \equiv \textbf{N}_j\dot{\textbf{q}_j}
\end{equation}
where \textbf{N}$_j$ and $\dot{\textbf{q}}_j$ are the corresponding 6\emph{n}$_j\times$\emph{n}$_j$ natural orthogonal complement (NOC) matrix of the velocity constraints \citep{angeles1988dynamic} and the \emph{n}$_j$-dimensional vector of joint rates, respectively. This NOC is an orthogonal complement matrix to the velocity constraint matrix, denoted with \textbf{D}$_j$, that relates the twists of the interconnected links of the $j^{th}$ tree-type subsystem  in the form of \textbf{D}$_j$\textbf{t}$_j$ = \textbf{0}, i.e., \textbf{D}$_j$\textbf{N}$_j\dot{\textbf{q}}_j$ $\equiv$ \textbf{0} \citep{sahaphdthesis}. Since \textbf{w}$^C_j$ = \textbf{D}$_j^T\mbox{\boldmath$\lambda$}^c_j$, where \mbox{\boldmath$\lambda$}$^c_j$ is the set of Lagrange multipliers representing the reactions at the uncut joints of the subsystem \emph{j}, it can then be easily shown, \textbf{N}$^T_j$\textbf{D}$^T_j\mbox{\boldmath$\lambda$}^c_j$ = \textbf{0} \citep{angeles1997fundamentals}. This result will be used next to eliminate the constraint wrenches. The NOC matrix \textbf{N}$_j$ for the $j^{th}$ subsystem when written in the decoupled form \citep{saha1999dynamics}, it leads to the well known concept of the DeNOC matrices, denoted as (\textbf{N}$_l)_j$ and (\textbf{N}$_d)_j$. Equation \ref{eqn:twist} can then be re-written as 
\begin{equation}
\label{eqn:twistdenoc}
\textbf{t}_j \equiv (\textbf{N}_l)_j(\textbf{N}_d)_j\dot{\textbf{q}}_j
\end{equation} 
where \textbf{N}$_j \equiv$ (\textbf{N}$_l$)$_j$(\textbf{N}$_d$)$_j$. Its time derivative is then given by
\begin{equation}
\label{eqn:twistratedenoc}
\dot{\textbf{t}}_j \equiv (\textbf{N}_l)_j(\textbf{N}_d)_j\ddot{\textbf{q}}_j + [(\dot{\textbf{N}}_l)_j(\textbf{N}_d)_j + (\textbf{N}_l)_j(\dot{\textbf{N}}_d)_j]\dot{\textbf{q}}_j
\end{equation} 
where (\textbf{N}$_l)_j$ and (\textbf{N}$_d)_j$ are the lower block triangular and block diagonal DeNOC matrices for the $j^{th}$ tree-type subsystem, respectively. Pre-multiplying Eq. \ref{eqn:uncons_eom} by the transpose of the DeNOC matrices, i.e., (\textbf{N}$_d^T)_j$(\textbf{N}$_l^T)_j$, yields a minimal set of equations of motion for the $j^{th}$ tree-type subsystem eliminating the constraint wrenches \citep{chaudhary2008dynamics}, i.e.,
\begin{equation}
\label{eqn:premult_eom}
(\textbf{N}_d^T)_j(\textbf{N}_l^T)_j(\textbf{M}_j\dot{\textbf{t}}_j + \mathbf{\Omega}_j\textbf{M}_j\textbf{E}_j\textbf{t}_j) = (\textbf{N}_d^T)_j(\textbf{N}_l^T)_j(\textbf{w$_j^E$ + w$_j^\lambda$})
\end{equation}
Note that the constraint wrenches at the uncut joints do not perform any work \citep{angeles1988dynamic}, as pointed out above after Eq. \ref{eqn:twist}. Hence, the term associated with the constraint wrenches, namely, \textbf{w}$_j^C$, vanishes from Eq. \ref{eqn:uncons_eom}. Now, substituting the expression of the generalized twist \textbf{t}$_j$ and its time derivative $\dot{\textbf{t}}_j$ from Eqs. \ref{eqn:twistdenoc} and \ref{eqn:twistratedenoc}, respectively, and rearranging the terms, one obtains the following constraint equations of motion for the $j^{th}$ tree-type subsystem in minimal order. They are expressed as
\begin{equation}
\label{eqn:const_eom}
\textbf{I}_j\ddot{\textbf{q}}_j + \textbf{C}_j\dot{\textbf{q}}_j = \mbox{\boldmath$\tau$}_j^E + \mbox{\boldmath$\tau$}_j^\lambda
\end{equation}
where the expressions of the generalized inertia matrix (GIM) \textbf{I}$_j$, matrix of convective inertia (MCI) terms \textbf{C}$_j$, and vectors of the generalized external forces \mbox{\boldmath$\tau$}$_j^E$, and the generalized force due to Lagrange multipliers \mbox{\boldmath$\tau$}$_j^\lambda$ are given as
\begin{equation}
\textbf{I}_j \equiv \textbf{N$_j^T$}\textbf{M}_j\textbf{N}_j,\hspace*{5pt} \textbf{C}_j \equiv \textbf{N$_j^T$} (\textbf{M}_j\dot\textbf{N}_j + \mathbf{\Omega}_j\textbf{M}_j\textbf{E}_j\textbf{N}_j), \mbox{\boldmath$\tau$}_j^E \equiv \textbf{N$_j^T$w$_j^E$},  \mbox{\boldmath$\tau$}_j^\lambda \equiv \textbf{N$_j^T$w$_j^\lambda$}
\end{equation}
where \textbf{N}$_j$ has been defined in Eq. \ref{eqn:twistdenoc}. Equation \ref{eqn:const_eom} represents the equations of motion for a cut-open tree-type subsystem subjected to Lagrange multipliers, and external wrenches (driving forces and torques). If $b$ = 1, i.e., there is only one branch, Eq. \ref{eqn:const_eom} regenerates to a set of equations for a serial-type system, as originally proposed by \citet{saha1999dynamics}. For the complete closed-loop system consisting of several such subsystems, i.e., $j = \{1, ..., v\}$, equations of motion in the form of Eq. \ref{eqn:const_eom} for each subsystem can be combined and represented in compact form as
\begin{equation}
\label{eqn:const_eom_sys}
\textbf{I}\ddot{\textbf{q}} + \textbf{C}\dot{\textbf{q}} = \mbox{\boldmath$\tau$}^E + \mbox{\boldmath$\tau$}^\lambda
\end{equation}
where the expression for each element in Eq. \ref{eqn:const_eom_sys} will be detailed in Sect. \ref{c5fd}. It should be noted here that similar minimal-order formulations also arise by using projection-based methods \citep{blajer1994projective}, Kane's method \citep{kane1985dynamics}, etc. by exploiting the relationships between the dependent and independent velocities and accelerations of the system under study.
\subsection{Loop-closure constraints}
\label{c5lcc}
%The equation of motion of all the subsystems together in the form of Eq. (\ref{eqn:const_eom_sys}) can be utilized for the solution of inverse dynamics problem, where the independent driving joint forces and torques along with the generalized forces due to Lagrange multipliers, \mbox{\boldmath$\tau$}$^\lambda$, are simultaneously solved using the approach described in \cite{blajer1}, \cite{nikravesh}. Unlike the inverse dynamics, in the forward dynamic analysis, the number of dynamic equations of motion available in Eq. (\ref{eqn:const_eom_sys}) are not sufficient to solve for both the joint accelerations and Lagrange multipliers. 
An additional set of equations resulting out of the kinematic constraints satisfying the loop-closures of a closed-loop system are used for the solution of $\ddot{\textbf{q}}$ and \mbox{\boldmath$\tau$}$^\lambda$. These equations relates the motion of the active and passive joints and are written at the acceleration level as
\begin{equation}
\label{eqn:lcc}
\mathbf{J\ddot{q}} = -\mathbf{\dot{J}\dot{q}} 
\end{equation}
where \textbf{J} is the constraint Jacobian matrix of the closed-loop system at hand. Depending on the number of independent kinematic loops and the suitable choice of the generalized coordinates, the sizes of \textbf{J} and $\ddot{\textbf{q}}$ will vary. It is worth noting that in Eq. \ref{eqn:const_eom_sys}, \mbox{\boldmath$\tau$}$^\lambda$ $\equiv$ \textbf{N$^T$w$^\lambda$} can also be represented as \mbox{\boldmath$\tau$}$^\lambda$ $\equiv$ \textbf{J$^T$}\mbox{\boldmath$\lambda$}, where the wrench \textbf{w}$^\lambda$ due to the Lagrange multipliers is the generalized constraint reaction force vector arising out of cutting the joints to make the closed-loop system open \citep{blajer1994projective,blajer2002elimination,blajer2011methods}.

Note that, using a definition of the generalized twist as \textbf{t} = [{\textbf{t}$_1^T$ \ldots \textbf{t}$_n^T$]$^T$, the loop-closure constraint equations can be expressed as $\hat{\textbf{J}}$\textbf{t} = \textbf{0}, where $\hat\textbf{J}$ is the corresponding \textit{m}$\times$6\textit{n}-dimensional constraint Jacobian matrix. The wrench \textbf{w}$^\lambda$ is then \textbf{w}$^\lambda$ = $\hat{\textbf{J}}^T\bm\lambda$. By projecting the dynamic equations, Eq. \ref{eqn:uncons_eom}, into the directions of $\dot{\textbf{q}}$ (by pre-multiplying them with \textbf{N}$^T$), one obtains \mbox{\boldmath$\tau$}$^\lambda$ $\equiv$ \textbf{N}$^T$\textbf{w}$^\lambda$ = \textbf{N}$^T\hat{\textbf{J}}^T\bm\lambda$ = \textbf{J}$^T\bm\lambda$, from which \textbf{J} = $\hat{\textbf{J}}$\textbf{N}. The same can also be deduced from the loop-closure constraints at the generalized twist level, i.e., $\hat{\textbf{J}}$\textbf{t} = \textbf{0}, which after using \textbf{t} = \textbf{N}$\dot{\textbf{q}}$ becomes $\hat{\textbf{J}}$\textbf{N}$\dot{\textbf{q}}$ = \textbf{J}$\dot{\textbf{q}}$ = \textbf{0}. Hence, \mbox{\boldmath$\tau$}$^\lambda$ $\equiv$ \textbf{J}$^T\bm\lambda$.
	\section{Inverse Dynamics}
	\label{c5invdyn}
	In inverse dynamics, for a given set of active joint trajectories, required joint torques are found. Since the closed-loop systems are cut-open, the knowledge of Lagrange multipliers is required, in order to use Eq. \ref{eqn:const_eom_sys} to solve for the joint torques  \mbox{\boldmath$\tau$}$^E$. For that Eq. \ref{eqn:const_eom_sys} is arranged as
	\begin{equation}
	\label{eqn:c5_inv_dyn_sys}
	\bm{\phi}^* = \left[ \begin{array}{cc}
	\textbf{1} & \hspace*{3pt}\textbf{J$^T$}\\
	\end{array} \right]
	\left[ \begin{array}{c}
	\mbox{\boldmath$\tau$}^E\\
	\mbox{\boldmath$\lambda$}
	\end{array} \right]
	\end{equation}
	where $\bm{\phi}^* \equiv \textbf{I}\mathbf{\ddot{q}} + \mathbf{C\dot{q}}$, which represents generalized inertia forces due to motion associated with $v$ subsystems. Moreover $\mathbf{1}$ is an appropriate matrix of ones and zeros according to the active and passive joint locations in a closed-loop system, while $\mathbf{J}$ denotes the constraint Jacobian matrix defined in Sect. \ref{c5lcc}. Moreover, $\mbox{\boldmath$\tau$}^E, \mbox{\boldmath$\lambda$}$ are the vectors of active joint torques and the Lagrange multipliers at the cut joints. Pre-multiplying the inverse of the coefficient matrix on the RHS of Eq. \ref{eqn:c5_inv_dyn_sys}, leads to the simultaneous solution of the joint torques and Lagrange multipliers, i.e., 
	\begin{equation}
	\left[ \begin{array}{c}
	\mbox{\boldmath$\tau$}^E\\
	\mbox{\boldmath$\lambda$}
	\end{array} \right] = \left[ \begin{array}{cc}
	\textbf{1} & \hspace*{3pt}\textbf{J$^T$}\\
	\end{array} \right]^{-1}\bm{\phi}^* 
	\end{equation}
In general, the coefficient matrix is of full-rank as long as the kinematic loops are independent and hence, inverse exists. However, the computational complexity to find the inverse of the matrix $\Big[\mathbf{1} \hspace{10pt} \mathbf{J}^T \Big]$ may be high, as the number of columns of \textbf{J} are equal to the active and passive joints in the closed-loop system. This strategy is simple but incurs higher-order computational complexity for the inverse dynamics solution.

Alternatively, a subsystem approach proposed in \citet{chaudhary2008dynamics} can be undertaken where the existence of small number of links and joints in each subsystem is exploited. In fact, the concept lead to the definitions of \emph{determinate} and \emph{indeterminate} subsystems. For the former, one can evaluate the joint forces and torques along with the Lagrange multipliers uniquely from its dynamic equations of motion, of Eq. \ref{eqn:const_eom}. However, for the \emph{indeterminate} subsystems, it is not possible. In such situation, inverse dynamics solution to a \emph{determinate} subsystem may make one or more indeterminate subsystems determinate. 
	%For instance, if a closed-loop system cut according to the above strategy results into $v$ subsystems, then the constrained equations of motion for each subsystem are written along with the Lagrange multipliers.
	%If the number of equations in a sub-system are equal to the number of unknowns, the sub-system is determinate. The same is  used for solving the unknown constraint forces and/or torques. In some cases the determinate sub-system has constraint forces only as the unknowns. In that case, these constrained forces are substituted into the indeterminate subsystem equations for the rest of unknowns, viz. forces/torques. 
	%\begin{eqnarray}
	%\bm{\phi}_1^* &=& \mbox{\boldmath$\tau$}_1^E + \textbf{J}_1^T\mbox{\boldmath$\lambda$}_1 \\\nonumber
	%\bm{\phi}_2^* &=& \mbox{\boldmath$\tau$}_2^E + \textbf{J}_2^T\mbox{\boldmath$\lambda$}_2 \\\nonumber
	%\vdots \\\nonumber
	%\bm{\phi}_j^* &=& \mbox{\boldmath$\tau$}_j^E + \textbf{J}_j^T\mbox{\boldmath$\lambda$}_j \\\nonumber
	%\vdots \\\nonumber
	%\bm{\phi}_v^* &=& \mbox{\boldmath$\tau$}_v^E + \textbf{J}_v^T\mbox{\boldmath$\lambda$}_v 
	%\end{eqnarray}
	%Some of the sub-systems according to this methodology either has a null torque vector, or the number of unknowns in the form of torque and Lagrange multiplier are equal to the number of links in that sub-system. Such subsystems are used to solve Lagrange multipliers first which are later substituted to other subsystems for unknown joint torques/forces. In the alternate case, the equation of the subsystem itself are used to solve for both Lagrange multipliers and joint torques simultaneously. 
This methodology although elegant, puts a restriction on the location of cut in a closed-loop system. In other words, some locations of cut may not result into a determinate subsystem, thereby making the methodology ineffective. Moreover, no straightforward analogy exists for the forward dynamics solutions.

Yet another approach which is proposed in this thesis, the attempt is to have a unified methodology for both the inverse as well as forward dynamics analyses. In this, a generalized joint constraint matrix (GJCM) of the closed-loop system under study is written first in terms of the loops and subsystems, i.e., 
\begin{equation}
\label{eqn:lcc_qdot}
\textbf{J$\dot{\textbf{q}}$} = \textbf{0} 
\end{equation}
In Eq. \ref{eqn:lcc_qdot}, matrix \textbf{J} is the desired GJCM whose block elements represented as \textbf{J}$_{i,j}$ correlate each subsystem with the loops and each loop with the subsystems. In case, an independent loop \textit{i} and a subsystem \textit{j} do not share any link \textbf{J$_{i,j}$} $\equiv$ \textbf{O}, i.e., the subsystem \emph{j} does not have any contribution to the formation of the $i^{th}$ loop. This observation is further illustrated using Figs. \ref{fig:c5escavator_systm_closed} and \ref{fig:c5escavator_subsystm_open}. In the figure, loop 1 (1-2-15-6-1) shares links \#1 and \#2 with subsystem I and link \#5 with subsystem II. It does not have any link common with subsystem III. Hence, according to the above observation, \textbf{J$_{1,1}$}, \textbf{J$_{1,2}$}, $\neq$ \textbf{O}, while \textbf{J$_{1,3}$} $\equiv$ \textbf{O}. Similarly loop 2 (7-14-13-12-7) is formed out of links \#5, \#6, \#8 and \#9 belonging to subsystem II and does not have any link common to subsystems I and III. Hence, \textbf{J$_{2,2}$} $\neq$ \textbf{O}, whereas \textbf{J$_{2,1}$}, \textbf{J$_{2,3}$} $\equiv$ \textbf{O}. Similarly, the others can be observed and the whole joint constraint matrix for the system shown in Fig. \ref{fig:c5escavator_systm_closed}, i.e., the excavator \citep{janssen2005multiloop}, can be given as
	\begin{equation}
	\label{eqn:J_escav}
	\mathbf{J}\equiv
	\left[ \begin{array}{ccc}
	\textbf{J$_{1,1}$} & \textbf{J$_{1,2}$} & \textbf{O} \\
	\textbf{O} & \textbf{J$_{2,2}$} & \textbf{O} \\
	\textbf{J$_{3,1}$} & \textbf{J$_{3,2}$} & \textbf{O} \\
	\textbf{J$_{4,1}$} & \textbf{O} & \textbf{J$_{4,3}$} \\
	\end{array} \right]
	\end{equation}
Now, for the \textit{i$^{th}$} independent loop, the loop-closure equations from Eq. \ref{eqn:lcc_qdot} could also be denoted as 
	\begin{equation}
	\label{eqn:Jidotq}
	\mbox{$\overline{\textbf{J}}_{i}$}\dot{\textbf{q}} = \textbf{0}
	\end{equation}
where \mbox{$\overline{\textbf{J}}_{i}$} $\equiv$ [\textbf{J}$_{i,1}$ \textbf{J}$_{i,2}$  ... \textbf{J}$_{i,v}$ ] and  $\dot{\textbf{q}}$ $\equiv$ [$\dot{\textbf{q}}_1^T$ $\dot{\textbf{q}}_2^T$ ... $\dot{\textbf{q}}_v^T$]$^T$. The latter denotes the time derivative of the generalized coordinates associated with \emph{v} subsystems. Physically, \mbox{$\overline{\textbf{J}}_{i}$} is defined as an entity that contains information about the relation of loop \emph{i} with various subsystems of the closed-loop system at hand. Similarly, \mbox{$\overline{\textbf{J}}_{j}$} $\equiv$ [\textbf{J}$_{1,j}^T$ \textbf{J}$_{2,j}^T$  ... \textbf{J}$_{u,j}^T$]$^T$ is defined as an entity that contains information about the relation of subsystem \emph{j} with various loops of the closed-loop system.

Now, differentiating Eq. \ref{eqn:lcc_qdot}, one can obtain the constraints at acceleration-level, which are given by Eq. \ref{eqn:lcc}. Denoting -\textbf{$\dot{\textbf{J}} \dot{\textbf{q}}$} as $\bm{\psi}$ for compact representation, Eq. \ref{eqn:lcc} is re-written as \textbf{J}$\ddot{\textbf{q}}$ = $\bm{\psi}$, where the matrix \textbf{J} and the vectors \textbf{$\dot{\textbf{q}}$} and $\bm{\psi}$ for a general closed-loop system are defined as 
	\begin{equation}
	\label{eqn:multiclose_gen_J}
	\textbf{J} \equiv
	\left[ \begin{array}{cccccc}
	\textbf{J$_{1,1}$} & \textbf{J$_{1,2}$} & \cdots & \textbf{J$_{1,j}$} & \cdots & \textbf{J$_{1,v}$} \\
	\textbf{J$_{2,1}$} & \textbf{J$_{2,2}$} & & & & \textbf{J$_{2,v}$} \\
	\vdots & & \ddots & & & \vdots \\
	\textbf{J$_{i,1}$} & & & \textbf{J$_{i,j}$} & & \textbf{J$_{i,v}$} \\
	\vdots & & & & \ddots & \vdots \\
	\textbf{J$_{u,1}$} & \textbf{J$_{u,2}$} & \cdots & \textbf{J$_{u,j}$} & \cdots & \textbf{J$_{u,v}$} \\
	\end{array} \right],\hspace*{5pt}
	\ddot{\textbf{q}} \equiv
	\left[ \begin{array}{c}
	\textbf{$\ddot{\textbf{q}}_{1}$}\\
	\textbf{$\ddot{\textbf{q}}_{2}$}\\
	\vdots\\
	\textbf{$\ddot{\textbf{q}}_{j}$}\\
	\vdots\\ 
	\textbf{$\ddot{\textbf{q}}_{v}$}\\
	\end{array} \right], \hspace*{5pt}
	\bm{\psi} \equiv
	\left[ \begin{array}{c}
	\mbox{$\bm\psi_1$}\\
	\mbox{$\bm\psi_2$}\\
	\vdots\\
	\mbox{$\bm\psi_i$}\\
	\vdots\\ 
	\mbox{$\bm\psi_u$}\\
	\end{array} \right] 
	\end{equation}
For a multi-closed-loop system, typically the matrix \textbf{J} of Eq. \ref{eqn:multiclose_gen_J} is sparse, as evident from Eq. \ref{eqn:J_escav}.

Next, the proposed inverse dynamics formulation is illustrated using a typical multi-body system in Fig. \ref{fig:c5escavatorpic}. The constrained dynamic equations of motion for each subsystem $j = \{1, 2 ,3\}$ can be written using the formulation provided in Sects. \ref{c5eomtreess} and \ref{c5lcc} as
	\begin{eqnarray}
	\label{eqn:esc_eom}
	\bm{\phi}_j^* = \mbox{\boldmath$\tau$}_j^E + \mbox{$\overline{\textbf{J}}_{j}$}\mbox{\boldmath$\lambda$}
	%\bm{\phi}_2^* &=& \mbox{\boldmath$\tau$}_2^E + \mbox{$\overline{\textbf{J}}_{2}$}\mbox{\boldmath$\lambda$} \\\nonumber
	%\bm{\phi}_3^* &=& \mbox{\boldmath$\tau$}_3^E + \mbox{$\overline{\textbf{J}}_{3}$}\mbox{\boldmath$\lambda$}
	\end{eqnarray}
	%\begin{equation}
	%%\bm{\phi}_1^* =
	%\left[ \begin{array}{c}
	%\Phi_1^*\\
	%\Phi_2^*\\
	%\Phi_3^*\\
	%\Phi_4^*\\
	%\end{array} \right]_1 = 
	%\left[ \begin{array}{c}
	%0\\
	%0\\
	%0\\
	%\tau_4^E
	%\end{array} \right]_1 + 
	%\left[ \begin{array}{c}
	%\mathbf{J}_{1,j}^T\\
	%\mathbf{J}_{2,j}^T\\
	%\mathbf{J}_{3,j}^T\\
	%\mathbf{J}_{4,j}^T\\
	%\end{array} \right]_1^T
	%\left[ \begin{array}{c}
	%\mbox{\boldmath$\lambda$}_1\\
	%\vdots\\
	%\mbox{\boldmath$\lambda$}_u\\
	%\end{array} \right]\\
	%\end{equation}
	where the torque vectors for the subsystems have the following structures:
	\begin{equation}
	\label{eqn:esc_tor}
	\mbox{\boldmath$\tau$}_1^E = 
	\left[ \begin{array}{c}
	0\\
	0\\
	0\\
	\tau_1^e
	\end{array} \right], \hspace{10pt}
	\mbox{\boldmath$\tau$}_2^E = 
	\left[ \begin{array}{c}
	0\\
	0\\
	0\\
	0\\
	\tau_2^e
	\end{array} \right], \hspace{10pt}
	\mbox{\boldmath$\tau$}_3^E = 
	\left[ \begin{array}{c}
	0\\
	\tau_3^e
	\end{array} \right]
	\end{equation}
	where $\tau_1^e$, $\tau_2^e$ and $\tau_3^e$ are the external actuation torques at joints 4, 13 and 10, shown in Fig. \ref{fig:c5escavator_systm_closed}. The vector dimensions in Eq. \ref{eqn:esc_tor} are based on the number of generalized coordinates that exist in each subsystem.
	%For an $i^{th}$ subsystem
	%\begin{equation}
	%\left[ \begin{array}{c}
	%\Phi_1^*\\
	%\vdots\\
	%\Phi_n^*\\
	%\end{array} \right]_i = 
	%\left[ \begin{array}{c}
	%0\\
	%\vdots\\
	%\tau_n^E\\
	%\end{array} \right]_i + 
	%\left[ \begin{array}{c}
	%	J_1 \\
	%	\vdots\\
	%	J_{u} \\
	%\end{array} \right]_i^T
	%\left[ \begin{array}{c}
	%	\mbox{\boldmath$\lambda$}_1\\
	%	\vdots\\
	%	\mbox{\boldmath$\lambda$}_u\\
	%\end{array} \right]\\
	%\end{equation}
	A key observation regarding the solution of inverse dynamics is made here. Since the DOF of the closed-loop system is less than the generalized coordinates, the equations corresponding to the unactuated joints would not contain $\tau^e$. Such equations are usually equal to the number of Lagrange multipliers. For instance, Eq. \ref{eqn:esc_eom} for the three subsystems amounts to eleven equations consisting of three external torque variables and eight Lagrange multipliers. Out of eleven equations, eight of them that do not contain torque variables on the RHS are combined together to obtain the Lagrange multipliers first. These Lagrange multipliers are then substituted into rest of the equations (three) for the determination of unknown torques as
	\begin{eqnarray}
	\label{eqn:c5_esc_eom2}
	\mbox{\boldmath$\tau$}_j^E = \bm{\phi}_j^* - \mbox{$\overline{\textbf{J}}_{j}$}\mbox{\boldmath$\lambda$}
	\end{eqnarray}
	The proposed inverse dynamics strategy does not put any restriction on the location of cut for solving a closed-loop system, which was required in subsystem approach of \cite{chaudhary2008dynamics} for obtaining an efficient inverse dynamics algorithm. The proposed strategy will be illustrated in Sect. \ref{c5examples} with the help of a 2-DOF pantograph and an 1-DOF carpet scrapping mechanism.
	\section{Forward Dynamics}
	\label{c5fd}
	In forward dynamics, given a set of input torques/forces, evaluate the joint trajectories. In general, the joint accelerations $\ddot{\textbf{q}}$, and the Lagrange multipliers \mbox{\boldmath$\lambda$}, for a closed-loop system are solved using the system of equations, namely, Eqs. \ref{eqn:const_eom_sys} and \ref{eqn:lcc}. In particular, the coefficients of Eq. \ref{eqn:const_eom_sys} for a closed-loop system are defined at subsystem level as \textbf{I} $\equiv$ \emph{diag} [\textbf{I$_1$}, \ldots, \textbf{I$_v$}]; 
	\textbf{C} $\equiv$ \emph{diag} [\textbf{C$_1$}, \ldots, \textbf{C$_v$}]; \textbf{q} $\equiv$ [\textbf{q$_1^T$}, \ldots, \textbf{q}$^T_v$]$^T$; \mbox{\boldmath$\tau$}$^E \equiv$ [(\mbox{\boldmath$\tau$}$_1^E$)$^T$, \ldots, (\mbox{\boldmath$\tau$}$_v^E$)$^T$]$^T$ and \mbox{\boldmath$\tau$}$^{\lambda} \equiv$ [(\mbox{\boldmath$\tau$}$_1^{\lambda}$)$^T$, \ldots, (\mbox{\boldmath$\tau$}$_v^{\lambda}$)$^T$]$^T$ --- \emph{v} being the number of subsystems. Moreover, the vectors, $\dot{\textbf{q}}$ and $\ddot{\textbf{q}}$, have the usual meaning of joint rate and joint acceleration vectors of \emph{v} subsystems. Equations \ref{eqn:const_eom_sys} and \ref{eqn:lcc} can then be combined as follows
	\begin{equation}
	\label{eqn:fwd_dyn_sys}
	\left[ \begin{array}{cc}
	\textbf{I} & \hspace*{3pt}\textbf{J$^T$}\\
	\textbf{J} & \textbf{O} \\
	\end{array} \right]
	\left[ \begin{array}{c}
	\textbf{$\ddot{\textbf{q}}$}\\
	-\mbox{\boldmath$\lambda$}
	\end{array} \right] = 
	\left[\begin{array}{c}
	\bm\varphi\\
	-\textbf{$\dot{\textbf{J}}\dot{\textbf{q}}$}\\
	\end{array} \right], \hspace*{5pt} \mathrm{where} \hspace*{5pt} \bm\varphi\\ \equiv \mbox{\boldmath$\tau$}^E - \textbf{C$\dot{\textbf{q}}$}
	\end{equation}
	where \mbox{\boldmath$\tau$}$^\lambda$ in Eq. \ref{eqn:const_eom_sys} is substituted with \textbf{J$^T$}\mbox{\boldmath$\lambda$} as discussed in the Sect. \ref{c5lcc}. Using Eq. \ref{eqn:fwd_dyn_sys}, vectors $\ddot{\textbf{q}}$ and \mbox{\boldmath$\lambda$} can be solved simultaneously by analytical/explicit inversion of the coefficient matrix that appear on the LHS, followed by their numerical integrations to obtain the joint motions. One may call this approach of forward dynamics as \emph{system approach}, which will be used hereafter. This approach is known to have disadvantages like constraint violation, inefficient calculations due to the inversion of a larger dimension matrix, etc. They are discussed in \citet{baumgarte1972stabilization, blajer2011methods}. Besides, this approach does not allow one to exploit the already available recursive forward dynamics algorithms and computer programs for the open-loop systems which may be very efficient and numerically stable \cite{mohan2007recursive}. For utilizing the algorithms of the open-loop systems, one needs to evaluate the Lagrange multipliers first and then treat them as external forces/torques to solve for the joint accelerations, from the  equations of motion of the individual subsystems. To achieve this, as proposed in \citet{baumgarte1972stabilization,blajer1994projective,stejskal1996kinematics}, vector $\ddot{\textbf{q}}$ must be expressed from Eq. \ref{eqn:const_eom_sys} as 
	\begin{equation}
	\label{eqn:c5_ddotq_SLM}
	\ddot{\textbf{q}} = \textbf{I$^{-1}$}(\textbf{J$^T$\mbox{\boldmath$\lambda$}} + \bm\varphi)
	\end{equation}
	Then, Eq. \ref{eqn:c5_ddotq_SLM} is substituted into Eq. \ref{eqn:lcc} to obtain \mbox{\boldmath$\lambda$} as
	\begin{equation}
	\label{eqn:c5_lambda_SLM}
	\mbox{\boldmath$\lambda$} = -\tilde{\textbf{I}}^{-1}(\textbf{J}\textbf{I$^{-1}$}\bm\varphi + \textbf{$\dot{\textbf{J}}\dot{\textbf{q}}$}), \hspace*{5pt} \mathrm{where} \hspace*{5pt} \tilde{\textbf{I}} \equiv \textbf{J}\textbf{I$^{-1}$}\textbf{J$^T$}
	\end{equation}
	For the numerical calculations, Eq. \ref{eqn:c5_lambda_SLM} needs to be used to solve for \mbox{\boldmath$\lambda$} first before they are treated as external forces/torques to obtain the joint accelerations, $\ddot{\textbf{q}}$ using Eq. \ref{eqn:c5_ddotq_SLM}. To find $\ddot{\textbf{q}}$, one may use the already existing algorithms for open- and tree-type systems. In many instances, recursive, efficient and stable algorithm are easily available for the open-loop systems that can be adopted immediately, thereby, improving the forward dynamics results of the closed-loop system under study. In fact, the expressions in Eq. \ref{eqn:c5_lambda_SLM} can also be obtained by block analytical inversion of the coefficient matrix given in Eq. \ref{eqn:fwd_dyn_sys}. This is shown in Appendix B. In Eq. \ref{eqn:c5_lambda_SLM}, it is evident that to compute the Lagrange multipliers, one requires the inversions of matrices \textbf{I} and $\tilde\textbf{I}$ which are \emph{n}$\times$\emph{n} and \emph{m}$\times$\emph{m} matrices, respectively. This approach is referred here as \emph{System-level Lagrange Multiplier} (SLM) approach.
	
	Note here that in \emph{system} and SLM approaches, one requires the inversions of larger size matrices. In order to avoid the inversions of large dimensional \textbf{I} and $\tilde\textbf{I}$, requiring $\mathcal{O}(n^3)$ and $\mathcal{O}(m^3)$ computations, a third alternative is proposed here, where the Lagrange multipliers are solved at subsystem level. Since subsystems have smaller dimensions and are usually serial/tree-type in nature compared to the complete closed-loop system, the resulting forward dynamics algorithm is expected to have both efficiency and numerical stability. One can refer this approach as \emph{Subsystem-level Lagrange Multiplier} (SSLM) approach. It is shown next how to find Lagrange multipliers using the SSLM approach. For that the constrained dynamics equations of motion for the $v$ subsystems are written from Eq. \ref{eqn:const_eom_sys} as
	\begin{equation}
	\label{eqn:c5_eom_fwd_dyn}
	\textbf{I}\mathbf{\ddot{q}} = \bm\varphi + \textbf{J}^T\mbox{\boldmath$\lambda$}, \hspace{5pt} \textnormal{where} \hspace{5pt}  \bm\varphi \equiv \mbox{\boldmath$\tau$}^E - \textbf{C}\mathbf{\dot{q}}
	\end{equation}
	and the matrix \textbf{I} and the vectors $\bm\varphi$, $\mbox{\boldmath$\lambda$}$ are defined as
	\begin{equation}
	\label{eqn:Ipsilamdadef}
	\textbf{I} \equiv
	\left[ \begin{array}{cccc}
	\textbf{I$_{1}$} & & & \textbf{O}'s\\
	&\textbf{I$_{2}$} & & \\
	& & \ddots & \\
	\textbf{O}'s & & & \textbf{I$_{v}$}\\
	\end{array}\right], \hspace{10pt}
	\bm\varphi \equiv
	\left[ \begin{array}{c}
	\bm\varphi_1\\
	\bm\varphi_2\\
	\vdots\\
	\bm\varphi_v\\
	\end{array} \right], \hspace{10pt}
	\mbox{\boldmath$\lambda$} \equiv
	\left[ \begin{array}{c}
	\mbox{\boldmath$\lambda$}_1\\
	\mbox{\boldmath$\lambda$}_2\\
	\vdots\\
	\mbox{\boldmath$\lambda$}_u\\
	\end{array} \right]
	\end{equation}
	In Eq. \ref{eqn:Ipsilamdadef}, \mbox{\boldmath$\lambda$}$_i$ denotes the vector of Lagrange multipliers associated with the \textit{i}$^{th}$ loop. These Lagrange multipliers are acting on the \textit{v}--subsystems and will be treated as external forces for the forward dynamics calculations of the open-loop subsystems. In order to detail the efficacy of the proposed approach, the expression for the joint accelerations of the \emph{j}$^{th}$ subsystem can be written using the matrices and vectors defined after Eq. \ref{eqn:lcc_qdot}, and in Eqs. \ref{eqn:Jidotq} and \ref{eqn:multiclose_gen_J}, i.e.,
	\begin{equation}
	\label{eqn:ddotqpropfwddyn}
	\textbf{$\ddot{\textbf{q}}$}_{j} = 
	\textbf{I$_j^{-1}$}(\bm\varphi_{j} + \mbox{$\overline{\textbf{J}}_{j}^T$}\mbox{\boldmath$\lambda$})
	= \hat{\bm\phi}_j + \mathbf{\hat{I}}_{j}^T\mbox{\boldmath$\lambda$} 
	\end{equation}
	where $\bm\varphi_j$, $\hat{\bm\phi}_j$ and $\hat{\textbf{I}}_{j}$ are defined as
	\begin{equation}
	\label{eqn:c5_upvarphi_propfwddyn}
	\bm\varphi_j \equiv \mbox{\boldmath$\tau_j$}^E
	- \textbf{C}_j\dot{\textbf{q}}_j,\hspace*{5pt}\hat{\bm\phi}_j \equiv
	\mathbf{I}_j^{-1}\bm\varphi_j,\hspace*{5pt}
	\hat{\textbf{I}}_{j} \equiv \mbox{$\overline{\textbf{J}}_{j}$}\mathbf{I}_j^{-T} \equiv [\hat{\textbf{I}}_{1,j}^T \hspace{5pt} \hat{\textbf{I}}_{2,j}^T \hspace{5pt} \ldots \hspace{5pt} \hat{\textbf{I}}_{u,j}^T] 
	\end{equation}
	in which $\hat{\textbf{I}}_{i,j}^T$ $\equiv$ \textbf{J}$_{i,j}$\textbf{I}$_j^{-T}$. Note that to obtain Eq. \ref{eqn:ddotqpropfwddyn}, one requires only the inversions of the generalized inertia matrices \textbf{I}$_j$'\emph{s} of the subsystems. Using Eqs. \ref{eqn:ddotqpropfwddyn} and \ref{eqn:c5_upvarphi_propfwddyn}, the joint accelerations for all the \emph{v} subsystems are then written in compact form as 
	\begin{equation}
	\label{eqn:c5_qddot_c5}
	\textbf{$\ddot{\textbf{q}}$} = \hat{\bm\phi} + \hat{\textbf{I}}^{T}\mbox{\boldmath$\lambda$}
	\end{equation}
	where $\hat{\bm\phi}$ and $\hat{\textbf{I}}$ are as follows:
	\begin{equation}
	\label{eqn:c5_phicapIcap}
	\hat{\bm\phi} \equiv 
	\left[ \begin{array}{c}
	\hat{\bm\phi}_1\\
	\hat{\bm\phi}_2\\
	\vdots\\
	\hat{\bm\phi}_v\\
	\end{array} \right]
	, \hspace*{5pt}
	\hat{\textbf{I}} \equiv
	\left[ \begin{array}{cccc}
	\hat{\textbf{I}}_{1,1} & \hat{\textbf{I}}_{1,2} & \cdots & \hat{\textbf{I}}_{1,v} \\
	\hat{\textbf{I}}_{2,1} & \ddots & & \vdots \\
	\vdots & & & \vdots \\
	\hat{\textbf{I}}_{u,1} & \cdots & \cdots & \hat{\textbf{I}}_{u,v} \\
	\end{array} \right]
	\end{equation}
	It is interesting to note that the matrix $\hat{\textbf{I}}$ has vanishing and non-vanishing block elements exactly at the same locations as those of \textbf{J} in Eq. \ref{eqn:multiclose_gen_J}. This will be evident from the examples later. Substituting the vector of the joint accelerations from Eq. \ref{eqn:ddotqpropfwddyn} into the constraints at acceleration level given after Eq. \ref{eqn:Jidotq} leads to 
	\begin{figure}[t!]
		\begin{center}
			\includegraphics[scale=1]{Chapter5/figures/flowchart.pdf}
			\caption{Proposed forward dynamics algorithm}
			\label{fig:c5flowchart} % Give a unique label
		\end{center}
	\end{figure}
	\begin{equation}
	\label{eqn:Ibarlamdaeqn}
	\overline{\textbf{I}}\bm\lambda = \mbox{$\overline{\bm\psi}$},\hspace*{5pt} \mathrm{where} \hspace*{5pt}\overline{\textbf{I}} \equiv \textbf{J}\hat{\textbf{I}}^T \hspace*{5pt} \mathrm{and} \hspace*{5pt}\mbox{$\overline{\bm\psi}$} \equiv \bm\psi - \textbf{J}\hat{\bm\phi}
	\end{equation}
	The block representations of the symmetric matrix $\overline{\textbf{I}}$ and vector $\mbox{$\overline{\bm\psi}$}$ are given next as follows
	\begin{equation}
	\label{eqn:c5_Ibarfwddyncl}
	\overline{\textbf{I}} \equiv
	\left[ \begin{array}{cccc}
	\overline{\textbf{I}}_{1,1} & \hspace*{5pt} & \hspace*{5pt} &  sym\\
	\overline{\textbf{I}}_{2,1} & \overline{\textbf{I}}_{2,2} & \hspace*{5pt} & \hspace*{5pt}\\
	\vdots & & \ddots & \hspace*{5pt} \\
	\overline{\textbf{I}}_{u,1}  & \cdots&  \cdots & \overline{\textbf{I}}_{u,u}  \\
	\end{array} \right], \hspace*{5pt}
	\mbox{$\overline{\bm\psi}$} \equiv  
	\left[ \begin{array}{c}
	\mbox{$\overline{\bm\psi}_1$}\\
	\mbox{$\overline{\bm\psi}_2$}\\
	\vdots\\
	\mbox{$\overline{\bm\psi}_u$}\\
	\end{array} \right]
	\end{equation}
	In Eq. \ref{eqn:c5_Ibarfwddyncl}, ``\emph{sym}'' means symmetric elements, whereas the block elements of the matrix $\overline{\textbf{I}}$ and vector \mbox{$\overline{\bm\psi}$} are represented as 
	\begin{equation}
	\label{eqn:c5_I_rsandPsi_cap}
	\overline{\textbf{I}}_{r,s} \equiv \sum_{j=1}^{v} \overline{\textbf{I}}_j^{rs} \hspace*{5pt} \mathrm{and} \hspace*{5pt} \mbox{$\overline{\bm\psi}_i$} \equiv \mbox{$\bm\psi_i$} - \sum_{j=1}^{v} \textbf{J}_{i,j}\hat{\bm\phi}_j, \hspace*{5pt} \mathrm{where} \hspace*{5pt} \overline{\textbf{I}}_j^{rs} \equiv \textbf{J$_{r,j}$}\textbf{I$^{-1}_j$}\textbf{J$_{s,j}^T$}\\
	\end{equation}
	where \emph{r}, \emph{s} = \{\emph{1}, ..., \emph{u}\} denote the row and column indices of matrix $\overline{\textbf{I}}$. The matrix $\overline{\textbf{I}}$ is referred here as \emph{Generalized Inertia Constraint Matrix} (GICM). Typically, each block element of the GICM $\overline{\textbf{I}}$, i.e., $\overline{\textbf{I}}_{r,s}$ has interesting interpretation. For instance, in a multi-closed-loop system, if loop \emph{i} is formed using subsystems \emph{j}$_1$ and \emph{j}$_2$, the corresponding block diagonal element of $\overline{\textbf{I}}$ in Eq. \ref{eqn:c5_Ibarfwddyncl} indicated with $\overline{\textbf{I}}_{i,i}$, will comprise of the matrices associated to the subsystems \emph{j}$_1$ and \emph{j}$_2$ only, i.e., $\overline{\textbf{I}}_{i,i}$ = $\overline{\textbf{I}}^{ii}_{j_1}$ + $\overline{\textbf{I}}^{ii}_{j_2}$. Similarly, if two loops \emph{i}$_1$ and \emph{i}$_2$ share a common link belonging to a subsystem \emph{j}, then one can obtain $\overline{\textbf{I}}_{i_1,i_2}$ $\equiv$ $\overline{\textbf{I}}^{i_1i_2}_j$. These interpretations will be much clearer in Sect. \ref{c5examples}.
	
	Using Eq. \ref{eqn:Ibarlamdaeqn}, the expressions  for the Lagrange multipliers are derived next by carrying out the block Gaussian elimination of $\overline{\textbf{I}}$, as shown in Appendix B for a particular case. For the proposed SSLM approach explained above, the calculation of the Lagrange multipliers would require only the explicit inversions of the subsystem inertia matrices \textbf{I}$_j$ and the intermediate matrices $\overline{\textbf{I}}_{r,s}$ only, which have generally much smaller dimensions compared to the complete tree-type system arising out of the original closed-loop system. As a result, opening of the original closed-loop system into several serial and tree-type systems was fully exploited. Moreover, the expressions for the vector of Lagrange multipliers corresponding to each independent loop are obtained in a \textit{reduced-order} form. This allows one to perform subsystem-level calculations instead of considering the whole system together, as done by other researchers \citep{bae1987recursive,blajer1994projective}. Furthermore, inversion of smaller-sized matrices is very advantageous in terms of the computational speed. In many instances they will have even explicit expressions, particularly, when the matrix sizes are 2$\times$ 2 or 3$\times$3, without requiring any numerical inversions that are prone to numerical sensitivity. This is one of the primary contributions of this work. Additionally, note that once vectors \mbox{\boldmath$\lambda$}$_i$'\emph{s} are known, one can solve for the forward dynamics of all serial- and/ or tree-type sub-systems using parallel computations, which is not discussed here due to possible diversion from the main focus of the present work.
	
For the solution of forward dynamics, i.e., to find the joint accelerations, one can now easily use already available well-established algorithms for open-loop systems e.g., \citet{hollerbach1980recursive,walker1982efficient,shah2013dynamics}, etc. In this thesis, the recursive dynamics algorithm \emph{ReDySim} for tree-type systems reported in \citet{shah2013dynamics} was used to solve the forward dynamics problem. It must be noted that the formulation proposed in \citet{shah2013dynamics} consists of several serial chain modules in a tree-type system which are denoted as modules. These modules are similar to the branches of the tree-type system proposed in our formulation. Figure \ref{fig:c5flowchart} shows the complete scheme of the proposed forward dynamics algorithm.
\section{Illustrations}
\label{c5examples}
In this section, the proposed inverse and forward dynamics formulations are illustrated with the help of a planar 2-DOF pantograph and a 1-DOF carpet scrapping mechanism. In particular, the calculation of the Lagrange multipliers at the \emph{reduced-order} level is shown before the DeNOC based software \textit{ReDySim} is used to perform the analyses.
\subsection{A 2-DOF pantograph}
\label{c52dofmech}
Figure \ref{fig:c5pantoskeltal} depicts the kinematic architecture of a five-bar pantograph mechanism used to develop the 2-DOF haptic device as part of this thesis. The mechanism has two degrees-of-freedom having four physical links which are numbered appropriately with the symbol `\#'. The ground link is of zero length for the advantages mentioned in Chapter 3. The system has only one planar kinematic loop. 
\begin{figure}[b!]
	\begin{center}
		\subfloat[Skeletal model]{\label{fig:c5pantoskeltal}\includegraphics[scale=0.55]{Chapter5/figures/pantograph1.pdf}} 
		\subfloat[RecurDyn model]{\label{fig:c5pantordyn}\includegraphics[scale=0.45]{Chapter5/figures/pantofig.pdf}} 
		\caption{2-DOF pantograph mechanism}
		\label{fig:c5pantograph}
	\end{center} 
\end{figure}

To perform the inverse and forward dynamics analyses, the system in Fig. \ref{fig:c5pantograph} was divided into two equivalent serial-chain subsystems by placing cut at joint 5. They are shown in Fig. \ref{fig:c5pantosubsystems} with each subsystem having two links each. Appropriate 2-dimensional Lagrange multipliers were added to the serial-chain subsystems that account for constraint forces at otherwise cut joints. Next, the constraint equations for the pantograph were derived from the planar kinematic loop or joints 1-2-5-4-3-1, as shown in Fig. \ref{fig:c5pantograph}. The constraint Jacobian \textbf{J} obtained by writing the loop-closure equations and the vector of joint-rates $\dot{\textbf{q}}$ are written as
\begin{equation}
\label{eqn:c5_Janddotq_panto}
\textbf{J} \equiv
\left[ \begin{array}{cc}
\textbf{J$_{1,1}$} & \textbf{J$_{1,2}$}  
\end{array} \right], \hspace*{5pt}
\dot{\textbf{q}} \equiv
\left[ \begin{array}{c}
\textbf{$\dot{\textbf{q}}_1$}\\
\textbf{$\dot{\textbf{q}}_2$}
\end{array} \right]
\end{equation}
where \textbf{J$_{1,1}$} and \textbf{J$_{1,2}$} are the 2$\times$2 matrices, respectively, whose elements are given in Appendix B. Also, $\dot{\textbf{q}}_1$ and $\dot{\textbf{q}}_2$ are the 2-dimensional vectors, respectively, which are defined as
\begin{equation}
\textbf{$\dot{\textbf{q}}_1$}\equiv
\left[ \begin{array}{c}
\dot{\theta_1}\\
\dot{\theta_2}\\
\end{array} \right], \hspace*{5pt}
\textbf{$\dot{\textbf{q}}_2$}\equiv
\left[ \begin{array}{c}
\dot{\theta_3}\\
\dot{\theta_4}\\
\end{array} \right]
\end{equation}
where $\theta_i$'s, for $i = \{1, 2, 3, 4\}$, are the joint angles of the revolute joints present in the 2-DOF pantograph. They are indicated in Fig. \ref{fig:c5pantosubsystems}.
\begin{figure}[t!]
	\begin{center}
		\subfloat[Subsystem I (two links)]{\label{fig:c5pantoa}\includegraphics[scale=1.2]{Chapter5/figures/pantographsub1.pdf}}\hspace{10pt}
		\subfloat[Subsystem II (two links)]{\label{fig:c5pantob}\includegraphics[scale=1.2]{Chapter5/figures/pantographsub2.pdf}} 
	\end{center}
	\caption{Subsystems of 2-DOF pantograph}
	\label{fig:c5pantosubsystems}
\end{figure}

For the inverse dynamics solution, vectors $\mbox{\boldmath$\tau$}^E$ and $\bm{\phi}^*$, corresponding to Eq. \ref{eqn:c5_inv_dyn_sys} for the two subsystems are given by
\begin{equation}
\bm{\phi}^* \equiv 
\left[ \begin{array}{c}
\bm{\phi}_1^*\\
\bm{\phi}_2^*\\
\end{array} \right], \hspace{10pt}
\mbox{\boldmath$\tau$}^E \equiv 
\left[ \begin{array}{c}
\mbox{\boldmath$\tau$}_1^E\\
\mbox{\boldmath$\tau$}_2^E\\
\end{array} \right]
\end{equation}
where the joint torque vectors for each subsystem are given by
\begin{equation}
\mbox{\boldmath$\tau$}_1^E \equiv 
\left[ \begin{array}{c}
\tau_1^E\\
0\\
\end{array} \right], \hspace{10pt}
\mbox{\boldmath$\tau$}_2^E \equiv 
\left[ \begin{array}{c}
\tau_3^E\\
0\\
\end{array} \right]
\end{equation}
where $\tau_1^E$, $\tau_3^E$ are the external actuator torques at joints 1 and 3, respectively, while joints 2 and 4 are passive in nature. Also, the associated Lagrange multiplier vector is 2-dimensional. Hence, the total number of unknown variables for the inverse dynamics of 2-DOF pantograph is four, i.e., $\tau_1^E$, $\tau_3^E$ and $\bm{\lambda}$. For the type of subsystems in Fig. \ref{fig:c5pantosubsystems}, one equation each corresponding to passive links are used for the solution of Lagrange multiplier first. The Lagrange multipliers are then substituted as external forces in Eq. \ref{eqn:c5_esc_eom2} for the calculation of required driving torque. 

Next, for the solution of forward dynamics, the constrained equations of motion for the two subsystems together can be expressed in the form of Eq. \ref{eqn:c5_eom_fwd_dyn}, in which matrix $\mathbf{I}$, and vector $\bm\varphi$ have the following representations
\begin{equation}
\label{eqn:c5_Iandpsipanto}
\textbf{I} \equiv 
\left[ \begin{array}{cc}
\textbf{I$_{1}$} & \emph{sym}\\
\textbf{O}'\emph{s} & \textbf{I$_{2}$} \\
\end{array} \right], \hspace*{5pt}
\bm\varphi \equiv
\left[ \begin{array}{c}
\bm\varphi_1\\
\bm\varphi_2\\
\end{array} \right]
\end{equation}
where $\bm\varphi_j$ $\equiv$ $\bm\tau_j^E$ - \textbf{C$_j$}\textbf{$\dot{\textbf{q}}_j$}, for j = \{1, 2\}. The associated constraint equations in Eq. \ref{eqn:c5_Janddotq_panto} are then expressed in the form of \textbf{J}$\ddot{\textbf{q}} = $\mbox{${\bm\psi}$}, where $\mbox{${\bm\psi}$}	\equiv -\dot{\textbf{J}}\dot{\textbf{q}}$ is expressed in block form as 
\begin{equation}
\mbox{${\bm\psi}$} \equiv
- \left[ \begin{array}{cc}
\textbf{\.J$_{1,1}$} & \textbf{\.J$_{1,2}$} \\
\end{array} \right]
\left[ \begin{array}{c}
\textbf{$\dot{\textbf{q}}_1$}\\
\textbf{$\dot{\textbf{q}}_2$}\\
\end{array} \right] 
\end{equation}
The joint accelerations were then obtained using Eqs. \ref{eqn:c5_qddot_c5} and \ref{eqn:c5_phicapIcap}, where 
\begin{equation}
\hat{\bm\phi} \equiv
\left[ \begin{array}{c}
\hat{\bm\phi}_{1} \\
\hat{\bm\phi}_{2} \\
\end{array} \right], \hspace*{5pt}
\mbox{\boldmath$\hat{\textbf{I}}$} \equiv 
\left[ \begin{array}{cc}
\mbox{\boldmath$\hat{\textbf{I}}$}_{1} & \mbox{\boldmath$\hat{\textbf{I}}$}_{2} \\
\end{array} \right]
\end{equation}
Since the 2-DOF pantograph has only one kinematic loop, matrix $\mathbf{\overline{I}}$ and vector \mbox{$\overline{\bm\psi}$} of Eq. \ref{eqn:c5_Ibarfwddyncl} are simply defined as 
\begin{equation}
\mbox{\boldmath$\overline{\textbf{I}}$} \equiv 
\left[ \begin{array}{c}
\mbox{\boldmath$\overline{\textbf{I}}$}_{1,1}
\end{array} \right], \hspace*{5pt} \textnormal{and} \hspace*{5pt}
\mbox{$\overline{\bm\psi}$} \equiv
\left[ \begin{array}{c}
\mbox{$\overline{\bm\psi}_1^{}$}
\end{array} \right]
\end{equation}
where the 2$\times$2 matrix \mbox{\boldmath$\overline{\textbf{I}}$}$_{1,1}$ is obtained as
\begin{equation} 
\label{eqn:c5_Ibar_panto}
\mathbf{\overline{I}}_{1,1} \equiv \mathbf{\overline{I}}_1^{11} + \mathbf{\overline{I}}_2^{11}   
\end{equation}
where the components in Eq. \ref{eqn:c5_Ibar_panto} are given by Eq. \ref{eqn:c5_I_rsandPsi_cap}. Moreover, the 2-dimensional vector \mbox{$\overline{\bm\psi}_1$} is given by
\begin{eqnarray}
\label{eqn:c5_Psibar_panto}
\mbox{$\overline{\bm\psi}_1^{}$} \equiv \mbox{${\bm\psi}_1^{}$} - \textbf{J$_{1,1}$}\hat{\bm\phi}_{1}-\textbf{J$_{1,2}$}\hat{\bm\phi}_{2}
\end{eqnarray}
Using the expressions in Eqs. \ref{eqn:c5_Ibar_panto} and \ref{eqn:c5_Psibar_panto}, the 2-dimensional vector \mbox{\boldmath$\lambda$} is simply obtained for the 2-DOF pantograph as
\begin{equation}
\label{eqn:c5_lamdafinal_rssr}
\mbox{\boldmath$\lambda$} = \mathbf{\overline{I}}_{1,1}^{-1}\mbox{$\overline{\bm\psi}_1$}
\end{equation}
Once the vector \mbox{\boldmath$\lambda$} was obtained, joint accelerations were calculated using Eq. \ref{eqn:c5_eom_fwd_dyn} by treating the two subsystems of Fig. \ref{fig:c5pantosubsystems} being subjected to external force \mbox{\boldmath$\lambda$}. Both inverse and forward dynamics results for the 2-DOF pantograph are presented in Sect. 5.5. Note that the expression in Eq. \ref{eqn:c5_lamdafinal_rssr} can also be obtained using the block Gaussian elimination of Eq. \ref{eqn:fwd_dyn_sys} for the 2-DOF pantograph as shown in Appendix B.
\subsection{An 1-DOF carpet scrapping mechanism}
\label{c5carpetscrapping}
In this specific example we illustrate the benefits of proposed algorithms in multi-loop systems which are of special interest in haptics applications also, for instance the 2- and 5-DOF closed-loop haptic interfaces \citep{stocco2001optimal}.
 \begin{figure}[h!]
 	\begin{center}
 		\subfloat[Architecture - kinematic diagram]{\label{fig:c5cs_arch}\includegraphics[scale=0.75]{Chapter5/figures/cs_arch.eps}} 
 		\subfloat[RecurDyn model]{\label{fig:c5cs_rdyn}\includegraphics[scale=0.45]{Chapter5/figures/cs_rdyn.pdf}}
 	\end{center}
 	\caption{A multi-loop carpet scrapping mechanism \citep{chaudhary2008dynamics}}
 	\label{fig:c5cs_mechanism}
 \end{figure} 
 
Figure \ref{fig:c5cs_arch} shows the line diagram of an 1-DOF industrial carpet scrapping mechanism used to clean a hand-knotted carpet \citep{chaudhary2008dynamics}. The mechanism is a suitable example of a multi-closed-loop system having seven links with the joints simply numbered as 1, 2, etc. To solve the inverse and forward dynamics problems, the system was divided into three subsystems by placing cuts at joints 8, 9 and 10. Lagrange multipliers were introduced at the cut joints to take care of the otherwise constraint forces/moments at those joints. Each Lagrange multiplier vector, say, at joint 8, consists of two components. The constraint equations for the complete closed-loop system were then derived by considering three independent loops, as shown in Fig. \ref{fig:c5cs_arch}, namely, loop 1 or joints 1-2-3-8-1, loop 2 or joints 2-3-9-5-4-2, and loop 3 or joints 5-6-7-10-5. These equations for the three subsystems and three independent loops in the form of Eq. \ref{eqn:lcc} have the following definitions for \textbf{J} and $\dot{\textbf{q}}$ 
	\begin{equation}
	\label{eqn:c5_Janddotq_cs}
	\textbf{J} \equiv
	\left[ \begin{array}{ccc}
	\textbf{J$_{1,1}$} & \textbf{J$_{1,2}$} & \textbf{O} \\
	\textbf{O} & \textbf{J$_{2,2}$} & \textbf{J$_{2,3}$} \\
	\textbf{O} & \textbf{O} & \textbf{J$_{3,3}$} \\ 
	\end{array} \right], \hspace*{5pt}
	\dot{\textbf{q}} \equiv 
	\left[ \begin{array}{c}
	\textbf{$\dot{\textbf{q}}_1$}\\
	\textbf{$\dot{\textbf{q}}_2$}\\
	\textbf{$\dot{\textbf{q}}_3$}\\ 
	\end{array} \right]
	\end{equation}
where, for the planar motion of the mechanism, \textbf{J$_{1,1}$}, \textbf{J$_{1,2}$}, \textbf{J$_{2,2}$}, \textbf{J$_{2,3}$} and \textbf{J$_{3,3}$} are the 2$\times$1, 2$\times$2, 2$\times$2, 2$\times$4 and 2$\times$4 matrices, respectively, whose scalar elements are given in Appendix B, while \textbf{$\dot{\textbf{q}}_1$},\textbf{$\dot{\textbf{q}}_2$} and \textbf{$\dot{\textbf{q}}_3$} are a scalar, 2- and 4-dimensional vectors, respectively, which are defined as 
	\begin{equation}
	\label{eqn:jointrates_cs}
	\dot{\textbf{q}}_1 \equiv \textit{$\dot\theta_1$}, \hspace*{5pt} \dot{\textbf{q}}_2 \equiv 
	\left[ \begin{array}{c}
	\textit{$\dot\theta_2$} \\
	\textit{$\dot\theta_3$} \\ 
	\end{array} \right] \mathrm{and} \hspace*{5pt}
	\dot{\textbf{q}}_3 \equiv 
	\left[ \begin{array}{c}
	\textit{$\dot\theta_4$} \\
	\textit{$\dot\theta_5$} \\ 
	\textit{$\dot\theta_6$} \\
	\textit{$\dot\theta_7$} \\ 
	\end{array} \right]
	\end{equation} 
where $\theta_i$'\emph{s}, for \emph{i} = \{1, 2, ..., 7\}, are the joint angles of the revolute joints present in the scrapping mechanism. They are indicated in Fig. \ref{fig:c5cs_subsystems}. Note that \textbf{O} in Eq. \ref{eqn:c5_Janddotq_cs} represents the matrix of zeros with compatible sizes depending upon where it appears. For example, \textbf{O} in the first block row of Eq. \ref{eqn:c5_Janddotq_cs} denotes the 2$\times$4 matrix of zeros, because there are two rows corresponding to the two scalar constraints of the first loop and four columns associated to the joint variables of the third subsystem. As mentioned earlier, several block matrices resulted from the non-sharing of any link between a loop and a subsystem must vanish. Hence, in Eq. \ref{eqn:c5_Janddotq_cs}, matrices \textbf{J$_{1,3}$}, \textbf{J$_{2,1}$}, \textbf{J$_{3,1}$} and \textbf{J$_{3,2}$} are matrices of zeros because loop 1 does not share any link from subsystem 3, and similarly for others. In Eq. \ref{eqn:jointrates_cs}, \textbf{$\dot{\textbf{q}}_1$}, \textbf{$\dot{\textbf{q}}_2$} and \textbf{$\dot{\textbf{q}}_3$} represent the vectors of joint-rates for the three subsystems, namely, subsystem 1, 2 and 3, respectively.
	\begin{figure}[t]
		\begin{center}
 			\subfloat[Subsystem I (one-link)]{\label{fig:c5cs_subsys1}\includegraphics[scale=1]{Chapter5/figures/cs_subsys1.pdf}} \hspace{5pt}
			\subfloat[Subsystem II (two-links)]{\label{fig:c5cs_subsys2}\includegraphics[scale=0.95]{Chapter5/figures/cs_subsys2.eps}}\hspace{5pt}
			\subfloat[Subsystem III (four-links)]{\label{fig:c5cs_subsys3}\includegraphics[scale=0.95]{Chapter5/figures/cs_subsys3.eps}} 
		\end{center}
		\caption{Subsystem-level representation of carpet scrapping mechanism}
		\label{fig:c5cs_subsystems}
	\end{figure}
Now, for the inverse dynamics solution, vectors $\mbox{\boldmath$\tau$}^E$, $\mbox{\boldmath$\lambda$}$ and $\bm{\phi}^*$ in Eq. \ref{eqn:c5_inv_dyn_sys}, corresponding to three subsystems of the scrapping mechanism are given by
	\begin{equation}
	\bm{\phi}^* \equiv 
	\left[ \begin{array}{c}
	\bm{\phi}_1^*\\
	\bm{\phi}_2^*\\
	\bm{\phi}_3^*\\
	\end{array} \right], \hspace{10pt}
	\mbox{\boldmath$\tau$}^E \equiv 
	\left[ \begin{array}{c}
	\mbox{\boldmath$\tau$}_1^E\\
	\mbox{\boldmath$\tau$}_2^E\\
	\mbox{\boldmath$\tau$}_3^E\\
	\end{array} \right], \hspace{10pt}
	\mbox{\boldmath$\lambda$} \equiv
	\left[ \begin{array}{c}
	\mbox{\boldmath$\lambda$}_1\\
	\mbox{\boldmath$\lambda$}_2\\
	\mbox{\boldmath$\lambda$}_3\\
	\end{array} \right]
	\end{equation}
where $\mbox{\boldmath$\tau$}_1^E = \tau_1^E$ is a scalar quantity representing the driving torque, while $\mbox{\boldmath$\tau$}_2^E$ and $\mbox{\boldmath$\tau$}_3^E$ are the 2- and 4-dimensional null vectors, respectively, as there are no external force or torque on sub-systems 2 and 3. Since the associated Lagrange multipliers are the 2-dimensional vectors, the total number of unknowns for the inverse dynamics analysis are seven, i.e., $\tau_1^E , \mbox{\boldmath$\lambda$}_1, \mbox{\boldmath$\lambda$}_2$ and $\mbox{\boldmath$\lambda$}_3$. Six scalar components of the Lagrange multiplier vectors are solved together using the six scalar constrained equations of motion for subsystems 2 and 3. The solution is obtained as
	\begin{equation}
	\label{eqn:c5_lamda_CS}
	\mbox{\boldmath$\lambda$} \equiv
	\left[ \begin{array}{cc}
	\textbf{J$_{1,2}$} & \textbf{O} \\
	\textbf{J$_{2,2}$} & \textbf{J$_{2,3}$} \\
	\textbf{O} & \textbf{J$_{3,3}$}
	\end{array} \right]^{-T}\left[ \begin{array}{c}
	\bm{\phi}_2^*\\
	\bm{\phi}_3^*\\
	\end{array} \right]
	\end{equation}
	The Lagrange multipliers from Eq. \ref{eqn:c5_lamda_CS} are then substituted in Eq. \ref{eqn:c5_esc_eom2} for the calculation of the required driving torque $\tau_1^E$ as
\begin{equation}
\tau_1^E = \phi_1^* - 
\textbf{J}^T_{1,1}\mbox{\boldmath$\lambda$} 
\end{equation}
This completes the analysis for the inverse dynamics of carpet scrapping mechanism.

Next, for the solution of forward dynamics, the constrained equations of motion for all the three subsystems are put in the form of Eq. \ref{eqn:c5_eom_fwd_dyn}, in which matrix \textbf{I}, and vector $\bm\varphi$ have the following representations
	\begin{equation}
	\label{eqn:c5_Iandpsi_cs}
	\textbf{I} \equiv 
	\left[ \begin{array}{ccc}
	\textbf{I$_{1}$} & & \textit{sym}\\
	&\textbf{I$_{2}$} & \\
	\textbf{O}'s & & \textbf{I$_{3}$} \\
	\end{array} \right], \hspace*{5pt} \bm\varphi \equiv  
	\left[ \begin{array}{c}
	\bm\varphi_1\\
	\bm\varphi_2\\
	\bm\varphi_3\\
	\end{array} \right]
	\end{equation} 
where $\bm\varphi_1$ $\equiv$ \mbox{\boldmath$\tau_1^E$} - \textbf{C$_1$}\textbf{$\dot{\textbf{q}}_1$}, $\bm\varphi_2$ $\equiv$ - \textbf{C$_2$}\textbf{$\dot{\textbf{q}}_2$} and $\bm\varphi_3$ $\equiv$ - \textbf{C$_3$}\textbf{$\dot{\textbf{q}}_3$}. The three associated constraint equations in the form of Eq. \ref{eqn:c5_Janddotq_cs} are then expressed as \textbf{J}$\ddot{\textbf{q}} = $\mbox{${\bm\psi}$}, where $\mbox{${\bm\psi}$} \equiv -\dot{\textbf{J}}\dot{\textbf{q}}$. Its representation in the form of block elements is given by, 
	\begin{equation}
	\mbox{${\bm\psi}$}\\
	\equiv
	\left[ \begin{array}{c}
	\mbox{${\bm\psi_1}$}\\
	\mbox{${\bm\psi_2}$}\\
	\mbox{${\bm\psi_3}$}\\
	\end{array} \right] \equiv - 
	\left[ \begin{array}{ccc}
	\textbf{\.J$_{1,1}$} & \textbf{\.J$_{1,2}$} & \textbf{O} \\
	\textbf{O} & \textbf{\.J$_{2,2}$} & \textbf{\.J$_{2,3}$} \\
	\textbf{O} & \textbf{O} & \textbf{\.J$_{3,3}$} \\ 
	\end{array} \right]
	\left[ \begin{array}{c}
	\textbf{$\dot{\textbf{q}}_1$}\\
	\textbf{$\dot{\textbf{q}}_2$}\\
	\textbf{$\dot{\textbf{q}}_3$}\\
	\end{array} \right] 
	\end{equation}
The joint accelerations can then be evaluated using Eq. \ref{eqn:c5_qddot_c5}, where the corresponding terms for the subsystems are as follows:
	\begin{equation}
	\label{eqn:c5_phicapIcap_cs}
	\hat{\bm\phi} \equiv 
	\left[ \begin{array}{c}
	\hat{\bm\phi}_{1} \\
	\hat{\bm\phi}_{2} \\
	\hat{\bm\phi}_{3} \\
	\end{array} \right], \hspace*{5pt} 
	\mathbf{\hat{I}} \equiv 
	\left[ \begin{array}{ccc}
	\mathbf{\hat{I}}_{1,1} & \mathbf{\hat{I}}_{1,2} & \textbf{O}\\
	\textbf{O} & \mathbf{\hat{I}}_{2,2} & \mathbf{\hat{I}}_{2,3}\\
	\textbf{O} & \textbf{O} & \mathbf{\hat{I}}_{3,3}\\
	\end{array} \right]
	\end{equation}
where $\hat{\bm\phi}_{i}$ and $\mathbf{\hat{I}}_{i,j}$ are defined in Eq. \ref{eqn:c5_upvarphi_propfwddyn} and after it, respectively. As pointed out after Eq. \ref{eqn:c5_phicapIcap}, the locations of the vanishing and non-vanishing block elements of the matrices \textbf{J} and $\mathbf{\hat{I}}$ appearing in Eq. \ref{eqn:c5_Janddotq_cs} and Eq. \ref{eqn:c5_phicapIcap_cs}, respectively, are same. This is due to the block diagonal representation of matrix \textbf{I} of Eq. \ref{eqn:c5_Iandpsi_cs} using which the matrix $\mathbf{\hat{I}}$ is derived. Further, for the scrapping mechanism the symmetric matrix $\mathbf{\overline{I}}$ and vector \mbox{$\overline{\bm\psi}$} of Eq. \ref{eqn:c5_Ibarfwddyncl} are obtained as
	\begin{equation}
	\label{eqn:c5_IhatPsihat_CS}
	\mathbf{\overline{I}} \equiv 
	\left[ \begin{array}{ccc}
	\mathbf{\overline{I}}_{1,1} & & \textit{sym}\\
	\mathbf{\overline{I}}_{2,1} & \mathbf{\overline{I}}_{2,2} & \\ 
	\textbf{O} & \mathbf{\overline{I}}_{3,2} & \mathbf{\overline{I}}_{3,3}\\
	\end{array} \right], \hspace*{5pt} \mbox{$\overline{\bm\psi}$} \equiv
	\left[ \begin{array}{c}
	\mbox{$\overline{\bm\psi}_1$}\\
	\mbox{$\overline{\bm\psi}_2$}\\
	\mbox{$\overline{\bm\psi}_3$}\\
	\end{array} \right]
	\end{equation}
	where the non-zero 2$\times$2 matrices $\mathbf{\overline{I}}_{1,1}$, $\mathbf{\overline{I}}_{2,1}$, $\mathbf{\overline{I}}_{2,2}$, $\mathbf{\overline{I}}_{3,2}$ and $\mathbf{\overline{I}}_{3,3}$ are obtained using Eq. \ref{eqn:c5_I_rsandPsi_cap} as 
	\begin{eqnarray} 
	\label{eqn:c5_componentsofIhat_CS}
	\mathbf{\overline{I}}_{1,1} \equiv \mathbf{\overline{I}}_1^{11} + \mathbf{\overline{I}}_2^{11},\hspace*{5pt} 
	\mathbf{\overline{I}}_{2,1} \equiv \mathbf{\overline{I}}_2^{21},\hspace*{5pt} \mathbf{\overline{I}}_{2,2} \equiv \mathbf{\overline{I}}_2^{22} + \mathbf{\overline{I}}_3^{22}, \mathbf{\overline{I}}_{3,2} \equiv \mathbf{\overline{I}}_3^{32}, \mathbf{\overline{I}}_{3,3} \equiv \mathbf{\overline{I}}_3^{33}
	\end{eqnarray}
	whereas the 2-dimensional vectors \mbox{$\overline{\bm\psi}$}$_1$, \mbox{$\overline{\bm\psi}$}$_2$ and \mbox{$\overline{\bm\psi}$}$_3$ are obtained using Eq. \ref{eqn:c5_I_rsandPsi_cap} as
	\begin{eqnarray}
	\label{eqn:c5_Psihat_CS}
	\mbox{$\overline{\bm\psi}_1$} &\equiv& \mbox{$\bm\psi_1$} - \textbf{J$_{1,1}$}\hat{\bm\phi}_{1}-\textbf{J$_{1,2}$}\hat{\bm\phi}_{2}, \hspace*{87pt} \nonumber\\
	\mbox{$\overline{\bm\psi}_2$} &\equiv& \mbox{$\bm\psi_2$} - \textbf{J$_{2,2}$}\hat{\bm\phi}_{2}-\textbf{J$_{2,3}$}\hat{\bm\phi}_{3}, \hspace*{5pt} 
	\mbox{$\overline{\bm\psi}_3$} \equiv \mbox{$\bm\psi_3$} - \textbf{J$_{3,3}$}\hat{\bm\phi}_{3} 
	\end{eqnarray}
Using the expressions of Eqs. \ref{eqn:c5_IhatPsihat_CS} - \ref{eqn:c5_Psihat_CS}, the 2-dimensional vectors of Lagrange multipliers \mbox{\boldmath$\lambda$}$_1$, \mbox{\boldmath$\lambda$}$_2$ and \mbox{\boldmath$\lambda$}$_3$ are obtained after carrying out the block Gaussian elimination of $\mathbf{\overline{I}}$ in Eq. \ref{eqn:c5_IhatPsihat_CS}, as explained after Eq. \ref{eqn:c5_I_rsandPsi_cap}. The expressions for \mbox{\boldmath$\lambda$}$_i$'\emph{s}, for \emph{i} = 3, 2, 1, are given below
	\begin{equation}
	\label{eqn:c5_lamda3_CS}
	\mbox{\boldmath$\lambda$}_3 \equiv \mathbf{\tilde{I}}_{3,3}^{-1}\Big(\mbox{$\overline{\bm\psi}_3$} - \mathbf{\overline{I}}_{3,2}\mathbf{\tilde{I}}_{2,2}^{-1}(\mbox{$\overline{\bm\psi}_2$} - \mathbf{\overline{I}}_{2,1} \mathbf{\overline{I}}_{1,1}^{-1}\mbox{$\overline{\bm\psi}_1$})\Big)\\
	\end{equation}
	\begin{equation}
	\label{eqn:c5_lamda2_CS}
	\mbox{\boldmath$\lambda$}_2 \equiv \mathbf{\tilde{I}}_{2,2}^{-1}(\mbox{$\overline{\bm\psi}_2$} - \mathbf{\overline{I}}_{2,1} \mathbf{\overline{I}}_{1,1}^{-1}\mbox{$\overline{\bm\psi}_1$} - \mathbf{\overline{I}}_{3,2}^T\mbox{\boldmath$\lambda$}_3)\\
	\end{equation}
	\begin{equation}
	\label{eqn:c5_lamda1_CS}
	\mbox{\boldmath$\lambda$}_1 \equiv \mathbf{\overline{I}}_{1,1}^{-1}(\mbox{$\overline{\bm\psi}_1$} - \mathbf{\overline{I}}_{2,1}^T\mbox{\boldmath$\lambda$}_2)\\
	\end{equation}
	where the 2-dimensional matrices $\mathbf{\tilde{I}}_{2,2}$ and $\mathbf{\tilde{I}}_{3,3}$ are given by
	\vspace{-10pt}
	\begin{equation}
	\label{eqn:c5_Itildadef_cs}
	\mathbf{\tilde{I}}_{2,2} \equiv \mathbf{\overline{I}}_{2,2} - \mathbf{\overline{I}}_{2,1}\mathbf{\overline{I}}_{1,1}^{-1}\mathbf{\overline{I}}_{2,1}^T \hspace*{15pt} \textnormal{and} \hspace*{15pt} \mathbf{\tilde{I}}_{3,3} \equiv \mathbf{\overline{I}}_{3,3} - \mathbf{\overline{I}}_{3,2}\mathbf{\tilde{I}}_{2,2}^{-1}\mathbf{\overline{I}}_{3,2}^T
	\vspace{-10pt}
	\end{equation}
Once the solutions for the vectors \mbox{\boldmath$\lambda$}$_1$, \mbox{\boldmath$\lambda$}$_2$, and \mbox{\boldmath$\lambda$}$_3$ were obtained, the joint accelerations were solved using Eq. \ref{eqn:c5_eom_fwd_dyn} by treating the three subsystems of Fig. \ref{fig:c5cs_subsystems} as three independent tree-type systems subjected to external forces \mbox{\boldmath$\lambda$}$_i$. 

In many algorithms, namely, those of recursive in nature, e.g., \emph{ReDySim} \citep{redysim}, the solutions of the joint accelerations require only $\mathcal{O}(n)$ computations -- \emph{n} being the DOF of the tree-type system at hand. Hence, computational benefits from two aspects, namely, the proposed reduced-order computation for $\bm\lambda$ requiring inversions of the smaller matrices corresponding to the subsystems, and the recursive $\mathcal{O}(n^3)$ computations for the joint accelerations using, say, \emph{ReDySim}, are expected. Some comparison on these aspects will be reported in Sect. \ref{c5resanddisc}.

One important observation in Eqs. \ref{eqn:c5_componentsofIhat_CS} and \ref{eqn:c5_Psihat_CS} is that the one based on the explanations given after Eqs. \ref{eqn:c5_Ibarfwddyncl} and \ref{eqn:c5_I_rsandPsi_cap}, i.e., the expressions of the GICM of Eq. \ref{eqn:c5_IhatPsihat_CS} contains information about the architecture of the closed-loop system at hand. For example, the (\emph{i},\emph{i}) block diagonal element holds information about the subsystems that are involved in the formation of loop \emph{i}. Referring to Fig. \ref{fig:c5cs_arch}, loop 1 is formed utilizing the links belonging to subsystems 1 and 2. Therefore, the block element at \{1,1\} is $\mathbf{\overline{I}}_1^{11} + \mathbf{\overline{I}}_2^{11}$. Similarly, loop 2 is formed using the links of subsystems 2 and 3. Hence, the block element at \{2,2\} is $\mathbf{\overline{I}}_2^{22} + \mathbf{\overline{I}}_3^{22}$. Since loop 3 is formed using subsystem 3 only, the block element at \{3,3\} is simplify $\mathbf{\overline{I}}_3^{33}$. The non-diagonal block elements could be deduced by noting if any link is shared between the two loops and the subsystem that carries the link. For example, in Fig. \ref{fig:c5cs_arch}, loops 1 and 2 share a common link \#2 belonging to subsystem 2. As a result, the block element at \{1,2\} is $\mathbf{\overline{I}}_2^{12}$, which is same as ($\mathbf{\overline{I}}_2^{21})^T$ because matrix $\mathbf{\overline{I}}$ is symmetric. Since loops 1 and 3 do not share any subsystem, the block element at \{1,3\} or \{3,1\} would be a null matrix. Similarly the block element at \{2,3\} or \{3,2\} can be obtained as $\mathbf{\overline{I}}_3^{23}$ = ($\mathbf{\overline{I}}_3^{32})^T$, because loops 2 and 3 share a common link \#5 belonging to subsystem 3. Hence, one can generate the complete GICM, $\mathbf{\overline{I}}$ of Eq. \ref{eqn:c5_Ibarfwddyncl} and Eq. \ref{eqn:c5_IhatPsihat_CS} using the subsystem representation of closed-loop architecture. This is a very powerful observation interlinking the interaction between the loops and the subsystems, which, as per the author's knowledge, was never reported in the literature. 

\section{Results and Discussions}
\label{c5resanddisc}
For generating required torque in the inverse dynamic analyses, a cycloidal trajectory was imparted at the active joints of the 2-DOF pantograph whereas a constant velocity was imparted at the active joint of carpet scrapping mechanism. The link parameters of the two mechanisms are given in Table \ref{tab1:parameters}. For the 2-DOF pantograph, the comparative plots of the inverse dynamics results in Fig. \ref{fig:c5panto_invdyn} (with and without gravity) confirm the correctness of the methodology presented in this thesis, whereas Fig. \ref{fig:c5cs_invdyn} confirms the same for the 1-DOF carpet scrapping mechanism. These results were compared with similar models developed in commercial software \textit{RecurDyn V8.2} \citep{recurdyn}.
\begin{figure}[t!]
		\begin{center}
			\subfloat[Without gravity]{\label{fig:c5panto_tor_wog}\includegraphics[scale=0.475]{Chapter5/figures/pantowog.eps}} 
			\subfloat[With gravity]{\label{fig:c5panto_tor_g}\includegraphics[scale=0.475]{Chapter5/figures/pantowg.eps}} 
		\end{center}
		\caption{Inverse dynamics results for the 2-DOF pantograph}
		\label{fig:c5panto_invdyn}
\end{figure}
\begin{figure}[t!] 
		\begin{center}
			\subfloat[Without gravity]{\label{fig:c5cs_tor_wog}\includegraphics[scale=0.45]{Chapter5/figures/csinvdynwog.eps}}
			\subfloat[With gravity]{\label{fig:c5cs_tor_wg}\includegraphics[scale=0.45]{Chapter5/figures/csinvdyntorg.eps}} 
		\end{center}
		\caption{Inverse dynamics results for the carpet scrapping mechanism}
		\label{fig:c5cs_invdyn}
\end{figure}
\begin{table}[t!]
		\centering
		\caption{Link parameters used for the dynamic analyses}
		\label{tab1:parameters} % Give a unique label
		\begin{tabular}{l|lll|lll}
			\hline
			&&&&&&\\
			& \multicolumn{3}{|c}{\textbf{2-DOF pantograph}} & \multicolumn{3}{|c}{\textbf{Scrapping mechanism}}\\
			\hline 
			&&&&&&\\
			\textbf{Sub-}&\textbf{Link}&\textbf{Length}&\textbf{Mass}&\textbf{Link}&\textbf{Length}&\textbf{Mass} \\
			\textbf{system}& \# & (m)& (kg)& \# & (m)& (kg)\\ [2 mm]
			\hline
			I & 1 & 0.1 & 0.0423 & 1 & 0.038 & 1.5 \\
			  & 2 & 0.1 & 0.0423 &   &       &      \\ 
			\hline
			II& 3 & 0.1 & 0.0423 & 2 & 0.1152 & 3  \\ 
			  & 4 &  0.1  & 0.0423 & 3 & 0.2304 & 5  \\ 
			\hline
			III & & & & 4 & 0.3346 & 4.2\\ 
			&  &  & & 5 & 0.239 & 3 \\
			& & & & 6 & 0.8365 & 10.5\\ 
			& & & & 7 & 0.239 & 3 \\ 
			\hline
		\end{tabular}
\end{table}

Simulation results were generated next by considering Lagrange multipliers as external forces and moments using the methodology presented in Sect. \ref{c5fd}. Input torques from inverse dynamics were provided at the actuated joints of the corresponding mechanisms. Figures \ref{fig:c5force_sim_results_panto_c5} and \ref{fig:c5force_sim_results_cs_c5} depict the simulation results for 2-DOF pantograph and 1-DOF carpet scrapping mechanism, respectively. The initial configurations of the mechanisms are given in Table \ref{tab2:initial_config}. The correctness of simulation results was judged from the deviation of actual joint angles from the desired ones shown in Figs. \ref{fig:c5pantoerror} and \ref{fig:c5cserror}. Since the deviations were very less (in the range of 10$^{-6}$), the simulation results produced by the proposed algorithm were considered correct. Note, however that the forced simulation results with gravity had larger deviation from the desired trajectory, as shown in Fig. \ref{fig:c5error_results_c5g}. This increased deviation is generally attributed to the phenomenon of zero eigenvalue instability, which is overcome using feedback control \citep{baumgarte1972stabilization}, as discussed in detail in \citet{khan2007modular}.
\begin{figure}[t!]
	\begin{center}
		\subfloat[Actuated joint angle variations]{\label{fig:c5pantoforced}\includegraphics[scale=0.485]{Chapter5/figures/pantowogfd.eps}}\hspace{1pt}
		\subfloat[Deviation from the desired $\theta_1$]  {\label{fig:c5pantoerror}\includegraphics[scale=0.485]{Chapter5/figures/pantowogerror.eps}}
	\end{center}
	\caption{Simulation results for the 2-DOF pantograph (g=0)}
	\label{fig:c5force_sim_results_panto_c5}
\end{figure}
\begin{figure}[t!]
	\begin{center} 
		\subfloat[Actuated joint angle variation - $\theta_1$]{\label{fig:c5csforced}\includegraphics[scale=0.485]{Chapter5/figures/cswogpos.eps}}\hspace{1pt}
		\subfloat[Deviation from the desired $\theta_1$]		{\label{fig:c5cserror}\includegraphics[scale=0.485]{Chapter5/figures/cserrorwog.eps}}
	\end{center} 
\caption{Forced simulation results for the carpet scrapping mechanism (g=0)}
\label{fig:c5force_sim_results_cs_c5}
\end{figure}
\begin{figure}[t!]
	\begin{center}
	\subfloat[Error in 2-DOF pantograph]{\label{fig:c5pantoerrorg}\includegraphics[scale=0.485]{Chapter5/figures/2doferrorwg.eps}}\hspace{1pt}
	\subfloat[Error in carpet scrapping mechanism]	{\label{fig:c5cserrorg}\includegraphics[scale=0.485]{Chapter5/figures/cserrorwg.eps}}
	\end{center}
\caption{Deviation of joint angles from desired trajectory (with gravity)}
\label{fig:c5error_results_c5g}
\end{figure}
	\begin{table}[b!]
		\centering
		\caption{Initial configuration parameters}
		\label{tab2:initial_config}
		\begin{tabular}{l|lll}
			\hline\\
			\textbf{System} & \textbf{Actuated} & \textbf{Initial} & \textbf{Initial}\\
			& \textbf{joint angles} & \textbf{position} (rad) & \textbf{velocity} (rad/s)\\
			\hline\\
			2-DOF pantograph & $\theta_1$, $\theta_3$ & $\pi$/2, 0 & 0, 0\\
			\hline\\
			Scrapping mechanism & $\theta_1$ & 0 & 10.472\\
			\hline		
		\end{tabular}
	\end{table}
	
In order to see the effect of the proposed methodology on the overall efficiency measured in terms of CPU times, two sets of simulations were attempted for the carpet scrapping mechanism. One with the conventional methodology based on \citet{baumgarte1972stabilization} and \citet{blajer1994projective}, where \mbox{\boldmath$\lambda$} was solved using Eq. \ref{eqn:c5_lambda_SLM} before $\ddot{\textbf{q}}$ was solved using Eq. \ref{eqn:c5_ddotq_SLM}, followed by the numerical integrations. Equations \ref{eqn:c5_ddotq_SLM} and \ref{eqn:c5_lambda_SLM} were solved using `$\backslash$' or \emph{mldivide} command of MATLAB software that uses Gaussian elimination \citep{MATLAB_mldivide} followed by forward and backward substitutions. Another one is the methodology based on Eqs. \ref{eqn:c5_lamda3_CS} - \ref{eqn:c5_lamda1_CS} and the recursive algorithm of \emph{ReDySim}. For each simulation, again two sets of parameters were used. One with a lower tolerance value for numerical integrations in MATLAB and larger simulation time, another for a higher tolerance value and smaller simulation time.
\begin{table}[b!]
\centering
\caption{CPU time comparison for simulation of scrapping mechanism} 
{(on Intel Core 2 Duo 2.2 GHz processor, using \emph{ode45} solver for integration)}\\
	\label{tab3:simulation_time}
\begin{tabular}{l|ll}
			\hline\\
			\textbf{Parameter} & \textbf{SLM} & \textbf{SSLM} \\
			\hline\\
			$\bm\lambda$ evaluation & 0.3045 sec$^a$ & 0.2205 sec$^a$\\
			\hline\\
			Overall simulation & 2.8 sec$^a$ & 2.72 sec$^a$\\
			& 6.51 sec$^b$ & 6.41 sec$^b$\\
			& 9.33 sec$^c$ & 9.25 sec$^c$\\
			\hline
\end{tabular}\\
\vspace*{5pt}
$^{a}$ free-sim. time = 15 sec, step size = 0.01 sec, abs. and rel. tolerance = 10$^{-3}$\\
$^{b}$ free-sim. time = 5 sec, step size = 0.01 sec, abs. and rel. tolerance = 10$^{-8}$\\
$^{c}$ forced-sim. time = 5 sec, step size = 0.01 sec, abs. and rel. tolerance = 10$^{-8}$\\
\end{table}

The average evaluation time for the Lagrange multipliers was estimated using the \emph{tic} and \emph{toc} commands of MATLAB, embedded inside the forward dynamics program. However, the overall CPU time was estimated by surrounding the \emph{ode45} function call with the \emph{tic} and \emph{toc} commands. Using the SLM and the SSLM approach, the overall CPU times for both the free- and forced-simulations and the times taken to calculate only the Lagrange multiplies as shown in Table \ref{tab3:simulation_time}. As seen from the CPU times, the proposed SSLM approach shows improvement. The average time taken to compute the Lagrange multipliers using the proposed SSLM approach was about 20 -- 30\% lesser compared to the SLM approach, as seen in the table. This improvement was mainly attributed to the reduced-order computations associated to the proposed SSLM approach. In terms of the overall CPU time, the improvement observed was however less. It was only around 1 -- 3\%. This is mainly attributed to the fact that the forward dynamics algorithm involves evaluation of various other quantities requiring MATLAB function calls, in addition to the calculation of the Lagrange multipliers and other dynamic quantities in \textit{ReDySim}. These function calls in MATLAB are known to have high overheads \citep{shampine1997matlab}, which take up most of the time for the forward dynamic calculations. Besides, the use of an integration solver in MATLAB, e.g., using \emph{ode45}, also contributes a considerable percentage of time into the overall simulation. Hence, 92 -- 94\% of the overall simulation time is spent in other calculations, whereas the evaluation of the Lagrange multipliers contribute only around 6 -- 8\% of the total computation time. Hence, the overall savings in CPU time due to the reduction in calculation time of Lagrange multipliers is about 6\% of 30\% (reduction in $\bm\lambda$ evaluation), i.e., 1.8\%, which is same as the average savings in the overall CPU time reported in Table \ref{tab3:simulation_time} for the simulation of carpet scrapping mechanism.

The proposed methodology thus has the following benefits; 1) Introduction of the definition of Generalized Inertia Constraint Matrix (GICM) in Eq. \ref{eqn:c5_Ibarfwddyncl}, providing important physical interpretations of the loop-subsystem interactions; 2) Improved efficiency in computing the Lagrange multipliers; and 3) Ability to use already existing algorithms for tree-type systems.
\section{Summary}
In this chapter, detailed dynamic modelling of  closed-loop multi-body systems was presented. A reduced order inverse dynamics formulation was proposed which does not restrict the place of cut in the subsystem based analysis. Further, a reduced-order forward dynamics formulation was proposed that enabled computationally efficient simulations. 
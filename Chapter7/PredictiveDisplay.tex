\chapter{Predictive Display}
In the last chapter it was shown using simulation results that time delay between the remote and local station leads the system towards in stability. In this chapter we use predictive display to alleviate the above problem.


Predictive display has been defined as using the computer for extrapolating the display forward in time \cite{sheridan}. In this a local model of the remote scene is used to predict and render the remote scene in response to operator command. It replaces the delayed video feedback with extrapolated synthesised  image of the remote environment and local enables the operator to perform the task normally. 

\section{Remote Scene Extrapolation} 
The visual data present in the current frame is the view that the robot has seen $h_1$ second earlier. In order to predict the current scene that the robot might be seeing we need to estimate the current position of the robot. Once the current position of the robot is known we re-construct a view form the old scene by moving the view point to the current position of the robot. To explain this further and to use it in the teleoperation simulation model discussed in chapter 6
\subsection{here}


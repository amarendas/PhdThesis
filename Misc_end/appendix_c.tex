\chapter{ }
\section{Optimal Design of Steering Linkages}
\label{Ch:ApenC}

Objective is to  minimize turning radius within the space available after installation  of the rear wheel assembly, battery compartment and scissor mechanism  on a given over all frame size of the vehicle. The design variables are $x,~ a, ~h,\text{and} ~w$. These variables are described in Chapter 2 and shown in Figures \ref{fig:steerCond} and \ref{fig:davis}


Objective function to be minimized is the turning radius, R  given by
\begin{equation}
R=\dfrac{a}{2}+\dfrac{w(h^2+b^2-bx)}{hx}
\end{equation}
where
\[\dfrac{2b}{h}= \dfrac{a}{w}\]
The constrain equations are 
\[
x<\frac{a}{2}-b-20
\]
\[ h^2+\left(\dfrac{a}{2}\right)^2<100^2
\]
The bonds on the design variables are given by
\[30<h<200\]
\[500<w<600\]
\[50<a<270\]
\[0<x\]
The minimization was carried out in Matlab using \textit{fmincon()} function.
 The resulting minimum radius thus obtained was 
 \[R_{min}=228 \text{ mm}\]
 The corresponding design variables obtained were 
 \[h=30 \text{ mm},~ w=515 \text{ mm}, ~ a=190 \text{ mm}, ~x=69.8 \text{ mm}\] 
 The achieved minimum turning radius was $R_{min}=415`mm$. This was due to introduction of support for the rack and links which were not taken into consideration in the optimization process.
 
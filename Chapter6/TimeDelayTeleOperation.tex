
\chapter{Simulation Time Delay Tele-operation}
\label{c7_DI_equimomental}
In tele-operation tasks, human operators observe a remote scene through cameras, while tele-manipulating the mobile robot. The operator response is based on the latest feedback images from the cameras.
\section{Tele operation architecture}
In this study standard kinematic model of the mobile platform is used. The inputs to the model are the left and right rear wheel velocities. Based on this velocity these velocities the front wheels are steered to satisfy the Ackerman condition.  The kinematic model of the platform is presented below

Where , and b=Distance between the rear wheels, VL and VR are the left and right wheel velocities. Since, the commands u1 and u2 reaches the remote platform instantly.
\section{Modeling the human operator}
\section{Simulation Results }
\section{Predictive Model based Feedback}
\subsection{using only dynamic model of vehicle}
\subsection{Kalman Filter with odometer data and Accelorometer?}


\section{Summary}
In this chapter, 
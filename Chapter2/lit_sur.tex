\setcounter{secnumdepth}{4} 
\chapter{Literature Review}
\label{c2_LitRev}

In this chapter we  discuss some of the important works published pertaining to the scope of this thesis. Few of the techniques and methods published in these literatures are directly used. The section on \textit{special purpose robot} list literatures which were reviewed to arrive at the over all design of the mobile robot. The dynamic analysis of the mobile platform was based on the work cited in section\textit{ Mathamatical modeling of Mobile Robot}. The literatures discussed in \textit{Path tacking algorithm}  section was used to arrive at the "human model" proposed in the thesis for simulation of teleoperation. The section on \textit{Teleoperation} discusses literatures in a much broader sense, such as force feedback, haptic interface  design etc., than what was adapted in this thesis. This was done for completeness of the subject. Last section on \textit{Predictive Display} though a part of human interface for teleoperation is discussed separately, because it forms one of the major component of the teleoperation system network designed  for the mobile robot  presented in this thesis.   
\section{ Special Purpose Mobile Robots}
This section discuss some of the special purpose and customized robots built for various typical applications. Design and fabrication of a low cost, solar powered mobile robot for  scientific missions on the Antarctic plateau is presented by Ray  \cite{ray2005design}. Honeycombe-glassfiber composite is used to provide high strength and low weight. Ohno, et al. \cite{ohno2011robotic}   developed a robotic control vehicle for measuring the radiation in the Fukushima Daiichi Nuclear Power Plant. Briones \cite{briones1994wall} presents a vacuum cup based wall climbing robot for inspection of  nuclear power plants. Galt \cite{galt1997tele} has developed eight legged teleoperated mobile robot for use in  nuclear industry. Development of magnetic wheel based mobile robot for painting of ship is discussed by Cho \cite{cho2013study}. Compliant link based mobile robot was designed and tested by Borenstein\cite{borenstein1995control}, in which author claims that due to its unique design, better dead-reckoning accuracy is achieved compared to other contemporary designs. This vehicle has two independent drive units or "trucks" that are free to rotate about a vertical shaft connected to the vehicle body. Each truck comprises two drive motors on a common axes and forms a differential drive system.Mechanical compliance is implemented by means of a linear bearing that allows relative motion between the front and rear truck. Other literatures giving details of  mobile robots based on   differential wheel, traction belt and omnidirectional wheel are given in the introductory part of chapter \ref{ch_3:Design}.

   
\section{Mathematical Modeling of a Mobile Robot}
A very comprehensive list of wheeled mobile robots using different wheel configuration is given by Muir and Neuman \cite{muir1987kinematic}. In this work kinematic equation of conventional, omnidirectional and ball wheels is presented.The kinematics of the WMR is derived by combining the kinematic information of   individual wheels . Detection of wheel slip based on  error in the least square solution is also discuses. Similar issues are addressed ny Alexander in \cite{alexander1989kinematics}, the major difference is that it uses physical friction model in there analysis in case rolling constrains are not satisfied in case of over actuated systems. A seminal work by Champion \cite{campion1996structural}, gives the structural classification of wheeled mobile robot based on the \textit{degree of mobility}, $\delta_m$, and \textit{degree of steeribility}, $\delta_s$, define there-in based on the number of conventional fixed wheels and  conventional centered orientable wheels. According to them any WMR fall in one of the 5 categories given by $(\delta_m,\delta_s)\rightarrow(3,0),(2,1),(1,1),(1,2)$. The configuration and posture kinematic models of each type is derived. Based on  dynamic model, the minimal number of actuators required for full maneuverability of each type is presented. Kinematic analysis of omi-directional over-actuated mobile robot  is presented in \cite{yi2002kinematics}. Two different method of forward kinematics is discussed along with  singularity analysis. Actuator switching scheme based on load distribution to avoid singularity is also described. 

Dynamic modeling of mobile manipulator can be categorized as: force based i.e the Newtaon-Eular (NE) formulation and  energy based as Eular-Lagrange (EL) . Hoostmans \cite{hootsmans1992motion} used NE method to arrive at  the dynamic model of mobile manipulator that has two link mounted on a mobile platform. Chung \cite{chung1998interaction} used EL method to arrive at the equations of motion for mobile manipulator. Geometric mechanics was used to adapt Luh and walker \cite{luh1980line} algorithm  by Boyer and Ali \cite{boyer2011recursive} for recursive inverse dynamics formulation for wheeled systems.   

Orthogonal compliment method utilizes the advantage of NE and EL approach to derive the equations of motion for a multibody dynamical system.  It uses the fact that motion can take place only in the null space of the constrain inducing matrix $A$ defined as $Ax=0$, where $x$ is a vector independent co-ordinates. The orthogonal compliment of the constraint inducing matrix $A$ is used to eliminate the non-working constraint  forces  and moments from the equations of motion.  Angles and Lee \cite{angeles1988formulation} used natural orthogonal compliment method to derive the equations of motion for holonomic mechanical system. In this,  orthogonal compliment was derived from the velocity constraints naturally, hence the name. This was  used by Angeles \cite{angeles2013fundamentals} and Saha in \cite{saha1989kinematics},\cite{saha1991dynamics} to derive equations of motion for WMR. 

\section{Path Tracking Algorithms}
Path tracing algorithm for control of mobile robots is used to arrive at the mathematical model of human operator for simulation of tele-operation loop in chapter 6. Geometry based path tracking algorithms are most intuitive and hence suitable our application. The major algorithms in this  category reported in literatures is  \textit{pure pursuit} \cite{coulter1992implementation}, \textit{follow the carrot} \cite{barton2001controller}, \textit{vector pursuit} \cite{wit2004autonomous}, and \textit{follow the past} \cite{hellstrom2006follow}. In pure-pursuit \cite{coulter1992implementation}, the steer angle of the robot is set so that the robot moves in circle to reach a \textit{ goal point} on the desired path. The goal point is based on the "Look Ahead Distance", which is practically the maximum distance one can see from the current vehicle position.  Corrective action is based on  position error of the vehicle,  orientation error is not taken into account explicitly. 

In case of "Follow the Carrot" method \cite{barton2001controller}  the steering angle is set proportional to the \textit{orientation error}. The orientation error is defined as the difference between the current orientation of the vehicle and the orientation required from the present position of the vehicle to reach the goal point on the referance path.  The proportionality constant is decide based on trial and error.

 The two previous geometric path tracking techniques only generate steering commands based upon the goal point on the reference trajectory to be traced. Hence, the requirement of vehicle posture control for accurate trajectory following remains unsatisfied. Even though, the path orientation and curvature are known at the reference point, an improved path tracking strategy such that the vehicle arrives at the reference point  with the correct orientation and curvature is suggested  by Wit in \cite{wit2004autonomous}. Wit uses Screw theory to find the error between the current Screw  and the required screw at the goal position.

Hellstrom \cite{hellstrom2006follow}, has proposed an algorithm which uses the knowledge of previously recorded steer angle, associated with  the path traced earlier. In this algorithm, the steer angle of the vehicle is set based on the orientation error, position error and the past recorded steer angle. A recent survey by Paden \cite{paden2016survey}, provides extensive review of other control strategies for path tracking of autonomous unmanned vehicles such as, those based on Lyapunov method, Model Predictive Controller, adaptive control etc. 


   
\section{Tele-operation}
Tele-operation deals with  connecting   human operator and the robot in order to reproduce human action at distance. Tele-operation is in general bidirectional or bilateral as the human needs to have a feedback in order to understand the results of his action and to perceive the remote environment. It started with its use in nuclear and space industries \cite{martin1985teleoperated,vertut1986teleoperations}, but now it is used in underwater exploration, surgery, live-power line maintenance, mining etc. all characterized by reducing the risk to human operators. The two major research areas are the "human interface" and "control" design.
\subsection{Human Interface}

Human interface is a means through which the operator interacts with the remote robot by perceiving the remote environment and sending commands accordingly. Thus, the human interface has two important purpose; one to excite the human senses to show the action of the executed task and to process the human command properly to execute it at the remote end.  Force and haptic feedback of remote environment drastically improves operators performance. A serial link haptic device PHANTOM \cite{massie1994phantom} was developed at MIT during 1994, to provide 3-DOF force feedback  for haptic(touch feedback) purpose. DELTA Haptice Device described in \cite{grange2001overview} provide 6-DOF force feedback with moderate force. Clover \cite{clover1997dynamic} has reported  use of off-the-shelf serial industrial robots for haptic realization of tasks requiring a large workspace and high force capability. Customized 10-DOF  haptic device is reported  for similar purpose in \cite{ueberle2004vishard10}. Design  of 6-DOF parallel mechanism for force feedback is discussed in \cite{yoon2001design}.

Another major form of human interface is the visual feedback, the main challenge is to provide depth perception of remote environment. Most stereoscopic systems used in telerobotics are based on shutter glasses \cite{aracil1997telerobotic,matthies1992stereo}, head-mounted displays \cite{matthies1992stereo} or polarized images \cite{hirzinger1994robots}. Systems based on shutter glasses hide user`s eyes alternately in synchronization to screen refreshment, which projects images for left and right eye alternately. A second type of interfaces is based on polarized images. The user is also required to wear glasses that filter the left and right images. The third type of interface are  the head mounted display such as "Google cardboard",  especially designed to immerse users into virtual environments, where the left and right images are projected to each eye using two separate screens or split screens.

\subsection{Control}
Control of tele-operetion system deals two issues, \textit{transparency} and \textit{stability}. Transparency deals with what information is to be exchanged between the remote and local station, so that the operator can have a natural feel of the remote environment. A position-position architecture is suggested by  Goertz \cite{goertz1961anl}, where  master position is passed as a command to the slave servo (position) controller, and slave position is returned to the master as a position command. A position-force architecture has been proposed by Flatau \cite{flatau1977sm} in this the master sends the position to the slave and the slave sends back the force felt by it in the remote environment. A general 4-channel architect has been suggested by Lawrence \cite{lawrence1993stability}, and transparency as  measure of performance in teleoperation has been defined and evaluated for different architectures.
 
An excellent survey article on control of bilateral teleoperation is given by Hokayem \& Spong \cite{hokayem2006bilateral}. Few of these are briefly presented here. A teleoperation system, comprised of a master and slave with their corresponding controllers, residing between the human operator and the environment, which  can be modeled as a two port network. Passivity based design of stabilizing control using  wave-variable concept and scattering theory has been proposed by Anderson and Spong \cite{anderson1989bilateral}, Rebelo \cite{rebelo2015time} and Anderson and Slotin \cite{niemeyer1991stable} etc.   Port-Hamiltonian  based approach has been used in \cite{stramigioli2010novel,stramigioli2005sampled}. Design of controller for time delayed systems  based on back-stepping method in combination with partial differential heat  equation is studied by  Kristic \cite{krstic2009delay}. 



\section{Predictive Display}
Delays are inherent in teleoperation over wireless network. Practically much of the delay is due to relay stations and limited  network bandwidth. As little as a half second delay in the visual feedback significantly reduces human performance \cite{chen2007human}. The operator tends to adopt an inefficient "move then wait and see" policy in order to complete the task.

    To over come performance deterioration of operator due to time delay in visual feedback two approaches have been reported in literature namely \textit{supervisory control or tele-assistance  } and \textit{predictive display}. In \textit{supervisory control} \cite{sheridan1986human,pook1994teleassistance,jagersand1995visual} the robot is partly guided by operator by giving the robot intermittent commands to achieve the goal. The drawback of such system is that operator looses direct contact of the task.
    In  predictive display systems, a natural and widely used techniques, synthesised view of the remote environment is displayed to the operator based on his movements. It has been used for space teleoperation as early as in 1993, as reported by Sherdan \cite{sheridan1993space}, Bejczy \cite{bejczy1990predictive} and Kim \cite{kim1993demonstration}. Where as the above two used a-prior modeling and  calibration of remote environment, Jagersand \cite{jagersand1999image}, used delayed visual feedback and operator control signal to build predicated image which was presented to the operator. The system was implemented with a fixed remote environment with a manipulator arm with  two wall mounted cameras. An estimation function was proposed 
    $I_i \approx \phi_k(x_i), i \in {1,..k}$, which approximates each image  $I_i$ seen so far on the trajectory ${x_1, x_2 .....x_k}$. Camera mounted on manipulator (eye-in-hand) based image predication method has been discussed using uncalibrated monocular camera by Yeres \cite{yerex2003predictive} and Deng \cite{deng2003predictive}. Multiple sensors based dense 3-D  map of remote  has been reported by Kelly \cite{kelly2011real} and \cite{burkert2004photorealistic}. While Kelly used fusion of  lidar camera data,  Burkert used stereo cameras. Hu \cite{hu2015line} has used SLAM based Predictive Dispalay  (PD)system for telemanipulation of mobile robot. In this approach texture and geometry of the remote site is transmitted instead of  video stream. This the author claims reduces bandwidth utilization.


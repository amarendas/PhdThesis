\setcounter{secnumdepth}{4} 
\chapter{Literature Review}
\label{c2_LitRev}

In this chapter, we  discuss some of the important works published pertaining to the scope of this thesis. Few of the techniques and methods published in these literatures are directly used. The section on \textit{Special Purpose Robots} lists literature which were reviewed to arrive at the overall design of the mobile robot developed in this research.
 The dynamic analysis of the mobile platform was based on the works cited in section\textit{ Mathamatical Modeling}. 
 The literature discussed in \textit{Path Tacking } helped to arrive at the "human model" proposed in this thesis for simulation of teleoperation. 
 The section on \textit{Tele-operation} discusses literature in a much broader sense; such as force feedback, haptic interface  design, etc., than what was adapted in this thesis. 
 This was done for completeness of the subject. 
 Last section on \textit{Predictive Display}, though a part of human interface for tele-operation is discussed separately because it forms one of the major components of the tele-operation system network designed  for the mobile robot  presented in this thesis.   
\section{ Special Purpose Robots}
This section reviews some of the special purpose  robots built for various typical applications. 
Design and fabrication of a low cost, solar powered mobile robot for  scientific missions on the Antarctic plateau was presented by Ray  \cite{ray2005design}. Honeycomb-glass-fibre composite was used to provide high strength and low weight. Ohno, et al. \cite{ohno2011robotic}   developed a robotic  vehicle shown in Figure \ref{fig:fukuRobot} for measuring the radiation in the Fukushima Daiichi Nuclear Power Plant. A survey of different types of climbing robots for non-destructive testing of pressure vessel is given by \cite{luk2006tele}, such as NERO III shown in Figure \ref{fig:nero3}.  Briones \cite{briones1994wall} presents a vacuum cup based wall climbing robot for inspection of  nuclear power plants. Galt \cite{galt1997tele} has developed eight-legged teleoperated mobile robot for the use in nuclear industry. Development of a magnetic-wheel based mobile robot for painting of ship is discussed by Cho \cite{cho2013study}. Compliant link based mobile robot was designed and tested by Borenstein\cite{borenstein1995control}, in which the author claims that due to its unique design, better dead-reckoning accuracy was achieved compared to other contemporary designs. This vehicle has two independent drive units or "trucks" that are free to rotate about a vertical shaft connected to the vehicle body. Each truck comprises two drive motors on a common axes and forms a differential drive system. Mechanical compliance was implemented by means of a linear bearing that allows relative motion between the front and rear truck. Other literature giving details of  mobile robots based on   differential wheel, traction belt and omnidirectional wheel are given in the introductory part of Chapter \ref{ch_3:Design}.
 \begin{figure}
	\centering
	\begin{minipage}{.5\textwidth}
		\centering
		\includegraphics[height=5cm,keepaspectratio]{Chapter2/fig/FukusimaRobot}
		\captionof{figure}{Robot for Fukushima Daiichi \\  Source:  Ohno, et al  \cite{ohno2011robotic} }
		\label{fig:fukuRobot}
	\end{minipage}%
	\begin{minipage}{.5\textwidth}
		\centering
		\includegraphics[width=1\linewidth,height=5cm,keepaspectratio]{Chapter2/fig/nero3}
		\captionof{figure}{NERO III \\ Source: Luk et al \cite{luk2006tele}}
		\label{fig:nero3}
	\end{minipage}
\end{figure}

   
\section{Mathematical Modeling}
A very comprehensive list of Wheeled Mobile Robots (WMR) using different wheel configuration is given by Muir and Neuman \cite{muir1987kinematic}. In the paper, kinematic equations of conventional, omnidirectional and ball wheels were presented. The kinematics of the WMR was derived by combining the kinematic information of   individual wheel. Detection of wheel slip based on  error in the least square solution was also discussed. Similar issues were addressed by Alexander in \cite{alexander1989kinematics}. The major difference is that he uses physical friction model in the analysis of over actuated systems where rolling constrains are not satisfied. A seminal work by Champion \cite{campion1996structural} gives the structural classification of wheeled mobile robots based on the \textit{degree of mobility}, $\delta_m$, and \textit{degree of steeribility}, $\delta_s$. It was based on the number of conventional fixed wheels and  conventional centered orientable wheels. According to them any WMR fall in one of the 5 categories given by $(\delta_m,\delta_s)\rightarrow(3,0),(2,1),(1,1),(1,2)$. 
The configuration and posture kinematic models of each type was derived. Based on  dynamic model, the minimal number of actuators required for full maneuverability of each type was presented. Kinematic analysis of omi-directional over-actuated mobile robot  was presented in \cite{yi2002kinematics}. Two different methods for forward kinematics was also discussed along with  singularity analysis. Actuator switching scheme based on load distribution to avoid singularity was also presented. 

Dynamic modeling of mobile manipulator can be categorized as: force based ,i.e, the Newtaon-Eular (NE) formulation and  energy based as in Eular-Lagrange (EL) equations. Hoostmans \cite{hootsmans1992motion} used NE method to arrive at  the dynamic model of a mobile manipulator that has two links mounted on a mobile platform. Chung \cite{chung1998interaction} used EL method to arrive at the equations of motion for a mobile manipulator. Geometric mechanics was used to adapt Luh and walker \cite{luh1980line} algorithm  by Boyer and Ali \cite{boyer2011recursive} to apply recursive inverse dynamics formulation to wheeled systems.   

Orthogonal compliment method utilizes the advantage of NE and EL approach to derive the equations of motion of  a multibody  system.  It uses the fact that the motion can take place only in the null space of the constrains inducing matrix $A$ defined as $Ax=0$, where $x$ is a vector of independent co-ordinates. The orthogonal compliment of the constraint inducing matrix $A$ is used to eliminate the non-working constraint  forces  and moments from the equations of motion.  Angeles and Lee \cite{angeles1988formulation} used the natural orthogonal compliment method to derive the equations of motion for holonomic mechanical systems. In this,  orthogonal compliment was derived from the velocity constraints naturally, hence the name. This was  used by Angeles \cite{angeles2013fundamentals} and Saha in \cite{saha1989kinematics},\cite{saha1991dynamics} to derive the equations of motion for a WMR. 

\section{Path Tracking}
Path tracing algorithm for the control of a mobile robot is used to arrive at the mathematical model of a human operator for simulation of tele-operation loop, as done in chapter 6. Geometry based path tracking algorithms are most intuitive and hence suitable for the present application. The major algorithms in this  category reported in the literatures are \textit{pure pursuit} \cite{coulter1992implementation}, \textit{follow the carrot} \cite{barton2001controller}, \textit{vector pursuit} \cite{wit2004autonomous}, and \textit{follow the past} \cite{hellstrom2006follow}. In pure-pursuit \cite{coulter1992implementation}, the steer angle of the robot is set so that the robot moves in circle to reach a \textit{ goal point} on the desired path. The goal point is based on the "Look Ahead Distance", which is practically the maximum distance one can see from the current vehicle position. The detailed discussion is given in Chapter \ref{c6_simulation}.  Corrective action was based on  position error of the vehicle, where orientation error was not taken into account explicitly. 

In case of "Follow the Carrot" method \cite{barton2001controller},  the steering angle is set proportional to the \textit{orientation error} defined in Figure \ref{fig:FollowCarrot}. The orientation error is defined as the difference between the current orientation of the vehicle and the orientation required from the present position of the vehicle to reach the goal point on the referance path.  The proportionality constant is decided based on trial and error.

 \begin{figure}
	\centering
	\begin{minipage}{.5\textwidth}
		\centering
		\includegraphics[height=5cm,keepaspectratio]{Chapter2/fig/FollowTheCarrot}
		\captionof{figure}{Follow the carrot}
		\label{fig:FollowCarrot}
	\end{minipage}%
	\begin{minipage}{.5\textwidth}
		\centering
		\includegraphics[width=1\linewidth,height=5cm,keepaspectratio]{Chapter2/fig/VectorPursuit}
		\captionof{figure}{Vector pursuit}
		\label{fig:VectorPursuit}
	\end{minipage}
\end{figure}
 The two previous geometric path tracking techniques  generate steering commands based upon the goal point on the reference trajectory to be traced. 
 %The requirement of vehicle posture control for accurate trajectory following remains unsatisfied.  
 Wit in \cite{wit2004autonomous} suggested a strategy to uses the path orientation and curvature  known at the goal point to improved path tracking,  such that the vehicle arrived at the goal point  with the correct orientation and curvature. Wit used Screw theory to find the error between the screw at the current location, point $A$  and the required screw at the goal position, Point $b$ as shown in Figure \ref{fig:VectorPursuit}. Control is then generated proportional to this error.

Hellstrom \cite{hellstrom2006follow} has proposed an algorithm which uses the knowledge of previously recorded steer angle, associated with  the path traced earlier. In this algorithm, the steer angle of the vehicle is set based on the orientation error, position error and the past recorded steer angle. A recent survey by Paden \cite{paden2016survey} provides extensive review of other control strategies for path tracking of autonomous unmanned vehicles such as those based on Lyapunov method, Model Predictive Controller, adaptive control, etc. 


   
\section{Tele-operation}
Tele-operation deals with  connecton of a   human operator with the robot in order to reproduce human action at distance. Tele-operation is in general bidirectional or bilateral as the human needs to have a feedback in order to understand the results of his action and to perceive the remote environment. It started with its use in nuclear and space industries \cite{martin1985teleoperated,vertut1986teleoperations}, but now it is used in underwater exploration, surgery, live-power line maintenance, mining, etc. All characterized by reducing the risk to human operators. The two major related research areas are the "human interface" and "control" design.
\subsection{Human interface}

Human interface is a means through which the operator interacts with the remote robot by perceiving the remote environment and sending commands accordingly. Thus, the human interface has two important purpose: one to excite the human senses to show the action of the executed task and to process the human command properly to execute it at the remote end.  Force and haptic feedback of remote environment drastically improves operator's performance. Hence a serial link haptic device PHANTOM \cite{massie1994phantom} was developed at MIT during 1994 to provide 3-DOF force feedback  for touch feedback purpose. DELTA Haptice Device described in \cite{grange2001overview} provides 6-DOF force feedback with moderate force. Clover \cite{clover1997dynamic} has reported  the use of off-the-shelf serial industrial robots for haptic realization of tasks requiring a large workspace and high force capability. Customized 10-DOF  haptic device was reported  for similar purpose in \cite{ueberle2004vishard10}. Design  of a 6-DOF parallel mechanism for force feedback is discussed in \cite{yoon2001design}.

Another major form of human interface is the visual feedback. The main challenge is to provide depth perception of remote environment. Most stereoscopic systems used in telerobotics are based on shutter glasses \cite{aracil1997telerobotic,matthies1992stereo}, head-mounted displays \cite{matthies1992stereo} or polarized images \cite{hirzinger1994robots}. Systems based on shutter glasses hide user's eyes alternately in synchronization to screen refreshment, which projects images for left and right eye alternately. A second type of interfaces is based on polarized images. The user is also required to wear glasses that filter the left and right images. The third type of interface is  the head mounted display such as "Google cardboard",  especially designed to immerse users into virtual environments where the left and right images are projected on each eye using two separate screens or split screens.

\subsection{Control}
Control of a tele-operetion system deals with two issues, \textit{transparency} and \textit{stability}. Transparency deals with what information is to be exchanged between the remote and local station so that the operator can have a natural feel of the remote environment. A position-position architecture is suggested by  Goertz \cite{goertz1961anl}, where  master position is passed as a command to the slave servo (position) controller, and slave position is returned to the master as a position command. A position-force architecture has been proposed by Flatau \cite{flatau1977sm} in which the master sends the position to the slave and the slave sends back the force felt by it in the remote environment. A general 4-channel architecture been suggested by Lawrence \cite{lawrence1993stability}, and transparency has been defined  as a measure of performance in teleoperation and evaluated for different architectures.
 
An excellent survey article on control of bilateral teleoperation was given by Hokayem and Spong \cite{hokayem2006bilateral}. Few of these are briefly presented here. A teleoperation system, comprised of a master and slave with their corresponding controllers, residing between the human operator and the environment,   can be modeled as a two port network. Passivity based design of stabilizing control using  wave-variable concept and scattering theory has been proposed by Anderson and Spong \cite{anderson1989bilateral}, Rebelo \cite{rebelo2015time} and Anderson and Slotin \cite{niemeyer1991stable}.   Port-Hamiltonian  based approach was used in \cite{stramigioli2010novel,stramigioli2005sampled}. Design of controller for time delayed systems  based on back-stepping method in combination with partial differential heat  equation was studied by  Kristic \cite{krstic2009delay}. 



\section{Predictive Display}
Delays are inherent in teleoperation over wireless network. Practically, much of the delay is due to relay stations and limited  bandwidth of the network.  As little as a half second delay in the visual feedback significantly reduces human performance \cite{chen2007human}. The operator tends to adopt an inefficient "move then wait and see" policy in order to complete the task.

    To overcome performance deterioration of the operator due to time delay in visual feedback, two approaches have been reported in the literature, namely, \textit{supervisory control or tele-assistance  } and \textit{predictive display}. In \textit{supervisory control} \cite{sheridan1986human,pook1994teleassistance,jagersand1995visual} the robot is partly guided by operator by giving the robot intermittent commands to achieve the goal. The drawback of such system is that operator looses direct contact with the task.
    In  predictive display systems, a natural and widely used techniques, synthesised view of the remote environment is displayed to the operator based on his movements. It has been used for space teleoperation as early as in 1993, which was reported by Sherdan \cite{sheridan1993space}, Bejczy \cite{bejczy1990predictive} and Kim \cite{kim1993demonstration}. Whereas the above two used a-prior modeling and  calibration of remote environment, Jagersand \cite{jagersand1999image} used delayed visual feedback and operator control signal to build predicated image which was presented to the operator. The system was implemented with a fixed remote environment with a manipulator arm with  two wall mounted cameras. An estimation function was proposed 
    $I_i \approx \phi_k(x_i), i \in {1,..k}$, that approximated each image  $I_i$ seen so far on the trajectory, i.e, ${x_1, x_2 .....x_k}$. Un-calibrated monocular camera mounted on manipulator (eye-in-hand) based image predication method was discussed   by Yeres \cite{yerex2003predictive} and Deng \cite{deng2003predictive}. Multiple sensors based dense 3-D  map of a remote scene was reported by Kelly \cite{kelly2011real} and \cite{burkert2004photorealistic}. While Kelly used fusion of  lidar and  camera,  Burkert used stereo cameras. Hu \cite{hu2015line} has used SLAM based Predictive Dispalay  (PD) system for telemanipulation of a mobile robot. In his approach, texture and geometry of the remote site was transmitted instead of  video stream. This, the author, claims reduces bandwidth utilization.
    \section{Summary}
    Different literatures relevant to the topic addressed in this research were presented. Each topic is a separate area of research in its self.  There are many more literatures in each topic which could be of great interest but was been skipped due to space constraints.  


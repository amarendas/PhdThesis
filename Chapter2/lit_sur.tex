\setcounter{secnumdepth}{4} 
\chapter{Literature Review}
\label{c2_LitRev}

This chapter presents the survey of the literature highlighting design  of different mobile robots and there  control either used for field operation  or for academic research test platforms. 

% Comment the below command to supress the random text
%\lipsum[3-20]
% % % % % % % % % % % % % % % % % % % % % % % % % % % % % % % % %

\section{Research Objectives}
Based on the surveyed literature the objective of the present research are listed below:

\begin{itemize}
%%\item[$\bullet$] Development of efficient method of extracting Denavit Hartenberg parameters from identified joint features, i.e., joint axis and center of rotation in the small workspace.
%%
%%\item[$\bullet$] Identification of kinematic chain of an installed manipulator and to incorporate it in the dynamic identification model. To include friction model in the dynamic identification for the two manipulators, one with very high GRR (KUKAKR5) and another robot KUKAiiwa equipped with F/T sensor at each joint.
%%
%%\item[$\bullet$] Dynamic identification using Decoupled Natural Orthogonal Complement (DeNOC) matrices based methodology for the recursive identification of the SIP.
%%
%%\item[$\bullet$] Application of equimomental formulation for the dynamic identification of the manipulators.
%%
%%\item[$\bullet$] To utilize recursive estimation technique based on Kalman filter approach for the dynamic identification of the manipulator.

\item Identify 
\item To 
\item To 
\item Deve
\item Es
\item To  

\end{itemize}


\section{Summary}
This chapter provided the 	

%%Lastly, with wide range of robot manufacturers today, and very few service providers, namely Nikon Metrology and Dynalog (both in USA) provides, identification and calibration. The efficient identification method with aptly acquired experimental data will be useful for the researchers and manufacturers to implement the identified model for control and simulation purpose. 
